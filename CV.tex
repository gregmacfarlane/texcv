\documentclass[margin,line]{res}

%This lets me make hyper references to BYU and Georgia Tech
\usepackage{etoolbox}
\usepackage{booktabs}
\usepackage{enumitem}
\usepackage{etaremune}
\usepackage[colorlinks=false]{hyperref}
\usepackage[table, dvipsnames]{xcolor}

% Define a macro to color a row of a table. https://tex.stackexchange.com/a/413008/12874
\makeatletter
\g@addto@macro{\endtabular}{\rowfont{}}% Clear row font
\makeatother
\newcommand{\rowfonttype}{}% Current row font
\newcommand{\rowfont}[1]{% Set current row font
\gdef\rowfonttype{#1}#1\ignorespaces%
}
\makeatother

% page numbers
\usepackage{fancyhdr}
\pagestyle{fancy}
\renewcommand{\headrulewidth}{0pt}
\renewcommand{\footrulewidth}{0pt}

\oddsidemargin -.5in
\evensidemargin -.5in
\textwidth=6.0in
\itemsep=0in
\parsep=0in

% if detailtrue, then the CV will show notes on publication venue
% as well as student evaluations for each course.
\newif\ifdetail
\ifcsdef{myvar}{\myvar}{}

%\detailtrue

%% ==========================================
%This allows me to use Bibtex to create my publication list
%\usepackage{natbib}
%\makeatletter
%\renewenvironment{thebibliography}[1]
%     {%\section*{\bibname}% <-- remove this to make section naming correct
%      \@mkboth{\MakeUppercase\bibname}{\MakeUppercase\bibname}%
%      \list{\@biblabel{\@arabic\c@enumiv}}%
%           {\settowidth\labelwidth{\@biblabel{#1}}%
%            \leftmargin\labelwidth
%            \advance\leftmargin\labelsep
%            \@openbib@code
%            \usecounter{enumiv}%
%            \let\p@enumiv\@empty
%            \renewcommand\theenumiv{\@arabic\c@enumiv}}%
%      \sloppy
%      \clubpenalty4000
%      \@clubpenalty \clubpenalty
%      \widowpenalty4000%
%      \sfcode`\.\@m}
%     {\def\@noitemerr
%       {\@latex@warning{Empty `thebibliography' environment}}%
%      \endlist}
%\makeatother

%\usepackage[resetlabels]{multibib}
%  \newcites{j,c,w}{{Test},{Test1},{Test2}}
% j- journals c-conferences p-presentations
\setdescription{font=\normalfont}

%% ==========================================
% This defines the list types and formatting
\newenvironment{list1}{
  \begin{list}{\ding{113}}{%
      \setlength{\itemsep}{0in}
      \setlength{\parsep}{0in} \setlength{\parskip}{0in}
      \setlength{\topsep}{0in} \setlength{\partopsep}{0in}
      \setlength{\leftmargin}{0.17in}}}{\end{list}}
\newenvironment{list2}{
  \begin{list}{$\bullet$}{%
      \setlength{\itemsep}{0in}
      \setlength{\parsep}{0in} \setlength{\parskip}{0in}
      \setlength{\topsep}{0in} \setlength{\partopsep}{0in}
      \setlength{\leftmargin}{0.2in}}}{\end{list}}

\newcounter{enuminitialize}

\newenvironment{myenum}[1][]
{%
 \setcounter{enuminitialize}{#1}
 \addtocounter{enuminitialize}{2}
 \begin{enumerate}[left= 4pt, itemsep=8pt, start=\value{enuminitialize}, label=\arabic*\addtocounter{enumi}{-2}]
}
{%
 \end{enumerate}
}


% define fonts
%\usepackage{fontspec}
%\setmainfont[Mapping=tex-text]{Palatino}
%\newfontface\acc{Marcellus SC}
\newcommand{\secfont}{\scshape }
\newcommand{\acc}{\scshape }

\begin{document}
% Center the name over the entire width of resume:
 \moveleft.5\hoffset\centerline{\LARGE\scshape Gregory S.  Macfarlane}
\vspace{.05in}
 \moveleft.5\hoffset\centerline{Brigham Young University}
 \moveleft.5\hoffset\centerline{
	 \href{mailto:gregmacfarlane@byu.edu}{gregmacfarlane@byu.edu}
   801.422.8505}
\vspace{.05in}
% address begins here
% Again, the address lines must be centered over entire width of resume:
 \moveleft.5\hoffset\centerline{430 Engineering Building}
 \moveleft.5\hoffset\centerline{Provo, UT 84602}

\begin{resume}

%\section{\sc Contact\\ Information}
\vspace{.05in}



%% ============================================================================
\section{\secfont Education}
\href{http://www.gatech.edu}{\acc Georgia Institute of Technology}
\\
	\vspace*{-.1in}
	\begin{list1}
	\item[] Ph.D., Transportation Systems Engineering  \hfill{May 2014}
%	\begin{list2}
%  	\item[] Advisor: Laurie A. Garrow
%		\item[] Dissertation: ``Using Big Data to Model Travel Behavior: Applications to Vehicle Ownership and Willingness-to-Pay for Transit Accessibility''
%%		\item[] GPA: 3.86/4.0
%%		\item[] Committee: Laurie A. Garrow (Chair - CEE), Juan Moreno-Cruz
%%(Economics), \\ Patricia L. Mokhtarian (CEE), Kari E. Watkins (CEE), Patrick S.
%%McCarthy (Economics), Jeffrey P. Newman (CEE)
%  \end{list2}
	\vspace*{.05in}
	\item[]M.S., Economics \hfill{May 2014}
	\end{list1}

\href{http://www.byu.edu}{\acc Brigham Young University}\hfill{December 2009}
\\
\vspace*{-.1in}
\begin{list1}
  \item[] B.S. with University Honors, Civil Engineering
	\item[] Minor	degrees in Mathematics and Asian Studies
%  \begin{list2}
%		\item[] Honors Thesis: ``Delay Patterns
%			and Perceptions at Free Right-turn Channelized Intersections''
%		\item[] Advisor: Mitsuru Saito
%%		\item[] GPA: 3.7/4.0
%    \item[] Studies Abroad: Nanjing, China (Engineering);
%			Naples, Italy (Arch{\ae}ology)
%	\end{list2}
\end{list1}

\noindent\makebox[\linewidth]{\rule{\linewidth}{0.4pt}}
%\textit{ Research Interests ---} Passive data and its applications in transport and
%land use modeling and forecasting.
%% ============================================================================
\section{\secfont Academic Experience}

{\acc Brigham Young University}

\vspace{-.4cm}
\textit{Assistant Professor} \hfill {November 2018 --- }\\

\vspace{-.4cm}
{\acc University of North Carolina, Chapel Hill}

\vspace{-.4cm}
\textit{Adjunct Lecturer/Teaching Assistant} \hfill {January 2017 --- May 2017}\\

\vspace{-.4cm}
{\acc Georgia Institute of Technology}

\vspace{-.4cm}
\textit{Post-doctoral Researcher} \hfill {January 2014 --- May 2014}\\

\vspace{-.4cm}
\section{\secfont Research\\ Interests}
Transportation planning and engineering, travel demand modeling,
passive transportation data, spatial and social correlation.

\noindent\makebox[\linewidth]{\rule{\linewidth}{0.4pt}}

%% ============================================================================
%\noindent\makebox[\linewidth]{\rule{\linewidth}{0.4pt}}
\section{\secfont Professional \\ Experience}

Registered professional engineer in North Carolina, license \#044518

{\acc Transport Foundry} Atlanta, Georgia

\vspace{-.3cm}
\textit{Transportation Engineer} \hfill {April 2017 --- October 2018}\\
Developed a data-driven travel demand model from passive data sources.

{\acc WSP | Parsons Brinckerhoff} Raleigh, North Carolina

\vspace{-.3cm}
\textit{Technical Principal, Systems Analysis Group} \hfill {June 2014 --- April 2017}\\
Developed advanced travel demand models for public sector clients.

\noindent\makebox[\linewidth]{\rule{\linewidth}{0.4pt}}




%% ==========================================
% BibTex reference management

% Journal Articles
%\section{\secfont Under \\ Review}
%\bibentry{epb_parkaccess}

%\noindent\makebox[\linewidth]{\rule{\linewidth}{0.4pt}}

\section{\secfont Refereed \\ Journal\\ Articles}
$^\dagger$indicates BYU graduate student authors, $^*$indicates BYU undergraduate authors.
\ifdetail 
Paper 1 came from my undergraduate work, papers 2 through 6 were from my doctoral research, and
papers 7 onward represent work completed during my time on the faculty at BYU. Number of citations
are from Google Scholar as of October 2024. Work published since my third year review appears in blue.\fi
\vspace{.3cm}

\begin{myenum}[19]






  \ifdetail {\color{NavyBlue} \fi

  \item \textbf{Macfarlane, G.S.}, Barnes, M.$^\dagger$, \& Gibbons, N.$^*$ (2024). 
  A utility-based approach to modeling systemic resilience of highway networks with an application in Utah.
  \textit{Journal of Transportation Engineering Part A: Systems}. 151 (1). \url{https://ascelibrary.org/doi/10.1061/JTEPBS.TEENG-8534}

  \item Wang, B.$^\dagger$, Fulda, N., Huang, Z.Y.$^*$, Schultz, G.G., \textbf{Macfarlane, G.S.}, Arnesen, J.$^*$, Khayyat, A.$^*$ (2024). 
  Predicting directional traffic volume at intersections with automated traffic signal performance measures data using machine learning algorithms.
 \textit{Transportation Research Record}. \url{https://doi.org/10.1177/03611981241252829}

  \item \textbf{Macfarlane, G.S.}, Riches, G.$^\dagger$, Youngs, E.K.$^\dagger$,
  Nielsen, J.A. (2024). Classifying location points as daily activities using
  simultaneously optimized DBSCAN-TE parameters.  \textit{Findings}.
  \url{https://doi.org/10.32866/001c.116197}

  \item Turley Voulgaris, C., \textbf{Macfarlane, G.S.}, Kaylor, J. (2024).
  Whose miles are these anyway? Estimating site-generated vehicle miles traveled.
  \textit{Journal of the American Planning Association}. 90(4), 593--609 \url{https://doi.org/10.1080/01944363.2023.2298962}

  \item  Wang, B.$^\dagger$, Schultz, G.G., \textbf{Macfarlane, G.S.}, Eggett, D.L., 
  \& Davis, M.C.$^*$ (2023) A methodology to detect traffic data anomalies in automated traffic 
  signal performance measures. \textit{Future Transportation}. 3(4), 1175-1194. \url{https://doi.org/10.3390/futuretransp3040064} 
  \ifdetail Citations: 2 \fi

  \item Daines, T.J.$^\dagger$, Schultz, G.G., \textbf{Macfarlane, G.S.}, \&
  Ward, C.$^*$ (2022). Evaluating real time ramp meter queue length estimation.
  \textit{Future Transportation}, 2(4), 807-827. \url{https://doi.org/10.3390/futuretransp2040045}
  \ifdetail Citations: 2 \fi

  \item \textbf{Macfarlane, G.S.}, Stucki, E.$^\dagger$, Redelfs, A.H., \& Spruance, L.A. (2022).
  Beyond proximity: utility-based access from location-based services data.
  \textit{International Journal of Environmental Research and Public Health}, 19(19), 12352.
  \url{https://doi.org/10.3390/ijerph191912352}.
  \ifdetail Citations: 1 \fi

  \item \textbf{Macfarlane, G.S.}, Turley Voulgaris, C., \& Tapia, T. (2022).
  City parks and slow streets: a utility-based access and equity analysis.
  \textit{Journal of Transport and Land Use}. 15(1): 587-612.
  \url{https://doi.org/10.5198/jtlu.2022.2009}
  \ifdetail Citations: 8 \fi

  \item Wang, B.$^\dagger$, Schultz, G.G., \textbf{Macfarlane, G.S.}, \& McCuen,
  S.$^*$ (2022). Evaluating signal systems using automated traffic signal
  performance measures. \textit{Future Transportation}. 2(3):  659-674.
  \url{https://doi.org/10.3390/futuretransp2030036}.
  \ifdetail Citations: 7 \fi
  \ifdetail }\fi
  \item \textbf{Macfarlane, G.S.}, Sheffield, M.H.$^\dagger$, Bennet, L.S.$^\dagger$, \& Schultz, G.G. (2021).
  The effect of transit signal priority on bus rapid transit headway adherence.
  \textit{Findings}. \url{https://doi.org/10.32866/001c.24499}.
  \ifdetail Citations: 2 \fi

  \item\textbf{Macfarlane, G.S.}, Hunter, C.$^*$, Martinez, A.$^*$, \& Smith, E.$^*$  (2021).
  Rider perceptions of an on-demand microtransit service in Salt Lake County, Utah
  \textit{ Smart Cities} 4(2): 717-727.  \url{https://doi.org/10.3390/smartcities4020036} 
  \ifdetail Citations: 15 \fi

  \item\textbf{Macfarlane, G.S.}, Boyd, N., Taylor, J.E., \& Watkins, K. (2021) Modeling the impacts of park access
on health outcomes: A utility-based accessibility approach. \textit{ Environment and
Planning B: Urban Analytics and City Science}, 48(8), 2289–2306. \url{https://doi.org/10.1177/2399808320974027}
  \ifdetail Citations: 24\fi

  \item Glenn, J., Bluth, M.$^*$, Christianson, M.$^*$, Pressley, J.$^*$, Taylor, A., \textbf{Macfarlane, G.S.}, \& Chaney, R. A. (2020).
  Considering the potential health impacts of electric scooters: an analysis of user reported behaviors in Provo, Utah.
  \textit{ International Journal of Environmental Research and Public Health}, 17(17), 6344. \url{https://doi.org/10.3390/ijerph17176344} 
  \ifdetail Citations: 62\fi

  \item\textbf{Macfarlane, G.S.}, Garrow, L.A., \& Moreno-Cruz, J. (2015). Do Atlanta
residents value MARTA? Selecting an autoregressive model to recover willingness
to pay. \textit{ Transportation Research Part A: Policy and Practice}, 78, 214–230.
\url{https://doi.org/10.1016/j.tra.2015.05.010} \ifdetail Citations: 9 \fi

  \item\textbf{Macfarlane, G.S.}, Garrow, L.A., \& Mokhtarian, P. L. (2015). The influences of
past and present residential locations on vehicle ownership decisions.
\textit{ Transportation Research Part A: Policy and Practice}, 74, 186–200.
\url{https://doi.org/10.1016/j.tra.2015.01.005} \ifdetail Citations: 52  \fi

  \item Brakewood, C., \textbf{Macfarlane, G.S.}, \& Watkins, K.E. (2015). The impact of
real-time information on bus ridership in New York City. \textit{ Transportation Research
Part C: Emerging Technologies}, 53, 59–75. \url{https://doi.org/10.1016/j.trc.2015.01.021} 
\ifdetail Citations: 203 \fi

  \item Binder, S., \textbf{Macfarlane, G.S.}, Garrow, L.A., \& Bierlaire, M. (2014).
Associations among household characteristics, vehicle characteristics and
emissions failures: An application of targeted marketing data. \textit{ Transportation
Research Part A: Policy and Practice}, 59, 122–133.
\url{https://doi.org/10.1016/j.tra.2013.11.005}\ifdetail Citations: 24 \fi

  \item Wall, T.A., \textbf{Macfarlane, G.S.}, \& Watkins, K.E. (2014). Exploring the use of
egocentric online social network data to characterize individual air travel
behavior. \textit{ Transportation Research Record}, 2400, 78–86.
\url{https://doi.org/10.3141/2400-09} \ifdetail Citations: 14  \fi

  \item McBride, J.H., Keach, R. W., Macfarlane, R.T., De Simone, G.F., Scarpati, C.,
Johnson, D.J., \textbf{Macfarlane, G.S.}, \& Weight, R.W.R. (2009). Subsurface visualization using
ground-penetrating radar for archaeological site preparation on the northern
slope of Somma-Vesuvius: a Roman site, Pollena-Trocchia, Italy. \textit{ Il Quaternario,
Italian Journal of Quaternary Sciences}, 22(1), 39–52. \url{https://portal.issn.org/resource/ISSN/0394-3356} \ifdetail Citations: 5  \fi
\end{myenum}

\ifdetail
\section{\secfont{Venue Notes}}
\begin{description}
  % get impact factor data from https://jcr.clarivate.com/jcr-cj/my-favorites
  % get cite score data from https://www.scopus.com/list/form/overview.uri?listTypeValue=Sources
  % get 
  \item[\textit{Transportation Research Part C: Emerging Technologies}] is a
  leading international journal with robust peer review focusing on applications
  and implications of technology in transportation systems.
  CiteScore: 15.8; 7/379 in civil engineering.  Impact Factor: 7.6. Scimago: Q1. Publisher: Elsevier.
  \item[\textit{Transportation Research Part A: Policy and Practice}] is
   a leading international journal with robust peer review focusing on
   transportation policy  analysis and the planning of transportation systems.
   CiteScore: 13.2; 15/379 in civil engineering. Impact Factor: 6.3. Scimago: Q1. Publisher: Elsevier.
  \item[\textit{Smart Cities}] is an international, scientific, peer-reviewed,
  open access journal on the science and technology of smart cities. CiteScore:
  11.2; 6/279 in urban studies. Impact Factor: 7.0. Scimago: Q1. Publisher: MDPI.
  \item[\textit{Journal of the American Planning Association}] is the quarterly
  journal of record for the planning profession. CiteScore: 11.0, 8/279 in urban studies. 
  Impact Factor: 3.3. Scimago: Q1.  Publisher: Taylor \& Francis.
  \item[\textit{Journal of Transport and Land Use}] is is the leading
  international journal that publishes original interdisciplinary papers on the
  interaction of transport and land use. CiteScore: 3.4; 75/279 in urban
  studies. Impact Factor: 1.6. Scimago: Q1. Publisher: University of Minnesota.
  \item[\textit{Environment and Planning B: Urban Analytics and City Science}]
  is a leading international journal with robust peer review publishing
  cutting-edge research in analytical methods for urban planning and
  design. CiteScore 6.1; 30/279 in urban studies. Impact Factor: 2.6. Scimago: Q1. Publisher: Sage.
  \item[\textit{International Journal of Environmental Research and Public Health}]
  is an interdisciplinary, open access journal with peer review. CiteScore: 7.3;
  104/665 in public health. Impact Factor: 4.6. Scimago: Q2. Publisher: MDPI.
  \item[\textit{Journal of Transportation Engineering Part A: Systems}] contains
  technical and professional engineering articles with robust peer review on the
  planning, design, construction, operation, and maintenance of air, highway,
  rail, and urban transportation systems and infrastructure.  CiteScore: 3.8;
  153/379 in civil engineering. Impact Factor: 1.8. Scimago: Q2. Publisher: ASCE.
  \item[\textit{Transportation Research Record}] is the Journal of the
  Transportation Research Board of the National Academies. Since 2020, the 
  journal has undergone a transformation to institute more rigorous peer review.
  CiteScore: 3.2; 181/379 in civil engineering. Impact Factor: 1.6. Scimago: Q2. Publisher: Sage.
  \item[\textit{Future Transportation}] an international, peer-reviewed, open
  access journal on the civil engineering, economics, environment and geography,
  computer science and other transdisciplinary dimensions of transportation. 
  CiteScore: 2.6; 95/204 in engineering. Impact Factor: the journal was established in 2019. Publisher: MDPI.
  \item[\textit{Findings}] is an interdisciplinary, independent, community-led,
  peer-reviewed, open access journal focused on short, clear, and pointed
  research results. The journal was established in 2019. Publisher: University of
  Sydney and McGill University.
\end{description}
\fi

% \textbf{Macfarlane, G.S.}, \& Lant, N.$^\dagger$ (2021). Identifying Segmentation Strategies in a Daily Activity Pattern Model for Wheelchair Users.

\noindent\makebox[\linewidth]{\rule{\linewidth}{0.4pt}}
\section{\secfont Selected \\ Peer- Reviewed Conference \\ Papers}
\vspace{.3cm}
\begin{myenum}[3]
  \ifdetail {\color{NavyBlue} \fi
\item Jarvis, D.L.$^\dagger$, \textbf{Macfarlane, G.S.}, Woolley, B.$^*$,
Schultz, G.G. (2024). Simulating incident management team response and
performance. \textit{Procedia Computer Science}. 238,
pp. 91-96.
\url{https://doi.org/10.1016/j.procs.2024.06.002}

\item Apelu, D.$^*$, \textbf{Macfarlane, G.S.}, Guthrie, W.S., Adams, N.$^*$, Mazzeo, B. (2023). Measuring Pavement Smoothness From the Perspective of E-Scooters. \textit{IEEE XPlore}, \url{https://doi.org/10.1109/IETC57902.2023.10152077}
\item \textbf{Macfarlane, G.S.}, Lant, N.$\dagger$ (2023). How Far Are We From Transportation Equity? Measuring the Effect of Wheelchair Use on Daily Activity Patterns. In: Antoniou, C., Busch, F., Rau, A., Hariharan, M. (eds) Proceedings of the 12th International Scientific Conference on Mobility and Transport. \textit{Lecture Notes in Mobility.} Springer, Singapore. \url{https://doi.org/10.1007/978-981-19-8361-0_10}
\ifdetail } \fi
\end{myenum}


\section{\secfont Selected\\ Reports}
\ifdetail These are technical reports completed under contract for the sponsoring agency;
each report was reviewed by a technical advisory committee prior to publication. Item 1 
resulted from postdoctoral activities at Georgia Tech, 
items 2 and 3 from my consulting practice, and items 4 through the present from my work 
since joining BYU.\fi
\vspace{0.3cm}
\begin{myenum}[3]
\item \textbf{Macfarlane, G.S.}, Atchley, S.H.$^\dagger$, Mansfield, K.A.$^*$, Baird, T. \& Gresham, C. (2024). 
  \textit{Activity-based Model Implementation and Analysis Considerations. }  (No. UT-24.16).  Utah Dept.\ of Transportation. Division of Research. \url{https://rosap.ntl.bts.gov/view/dot/77610}

  \item  \textbf{Macfarlane, G.S.}, Atchley, S.H.$^\dagger$ (2023).
  \textit{Identifying Microtransit Service Areas through Microsimulation}. 
  (No. UT-23.01). Utah Dept.\ of Transportation. Division of Research.
  \url{https://rosap.ntl.bts.gov/view/dot/66312}


\item Cruz, J., \textbf{Macfarlane, G.S.}, Xu, Y., Rodgers, M.O., \& Guensler,
  R. (2015). \textit{Sustainable Transportation Curricula}. National Center for
  Sustainable Transportation. \url{https://escholarship.org/uc/item/3c13q43c}.
\end{myenum}


\section{\secfont Conferences Organized}
\ifdetail
This includes conferences and symposia for which I served on an organizing or
scientific committee.
\fi
\vspace{0.3cm}

\begin{myenum}[1]
  \ifdetail {\color{NavyBlue} \fi
  \item{\textit{Activity-based Modeling Symposium} (2024). M\"ockel, R. (host),  Shaw, A., Bhat, C.R., Erhardt, G.D.,  \textbf{Macfarlane, G.S.}, \& Mokhtarian, P.L., scientific committee.   Hosted in Raitenhaslach, Germany with funding from the German Research Foundation. \url{https://www.mos.ed.tum.de/en/tb/workshops/abm2024/}}
  \ifdetail } \fi
\end{myenum}  

\section{\secfont Selected \\Presentations}
\ifdetail This includes invited presentations to academic and non-academic audiences, as
well as presentations resulting from abstract-only submission. Includes both
lectern sessions and posters. Item 1 came from my undergraduate honors thesis, items 2 through 4 from doctoral research,
items 5 through 10 from my work as a consultant, and items 11 through the present represent work
completed during my time at BYU.\fi

\vspace{0.3cm}



\begin{myenum}[15]
  \ifdetail {\color{NavyBlue} \fi
\item \textbf{Macfarlane, G.S.} (2024). A multiple modeling sandbox. In \textit{Activity-based models symposium}. Lectern presentation. Raitenhaslach, Germany.
\item Turley Voulgaris, C., Jensen, A.F., \textbf{Macfarlane, G.S.} (2024). Children’s mode choice and independence for the journey to school. In \textit{17th International Conference on Travel Behavior Research}. Lectern presentation. Vienna, Austria.
\item \textbf{Macfarlane, G.S.} Modeling with Big Data. (2023) At \textit{Technische Universit\"at M\"unchen}, invited lecture in Dr. Rolf M\"ockel's travel modeling course.
\item Guan, H.Z., Van Hentenryck, P., Erhardt, G.D., \textbf{Macfarlane, G.S.}, \& Watkins, K.E. (2023). Lessons from the design of on-demand multimodal transit systems in two cities. In \textit{Innovations in Transportation Analysis and Planning Conference}. Lectern presentation. Indianapolis, Indiana. \textit{Winner, best presentation by a student author.}
\item Ducuara, A. Holtrop-Kohl, L., Begay, S., Spruance, L., Redelfs, A., \textbf{Macfarlane, G.S.} (2023). Hungry for change: how cutting-edge research is helping to reduce food deserts in Utah. In \textit{Move Utah Summit}. Moderated panel discussion. Salt Lake City, Utah.
\item Singh, G., Young, S., \textbf{Macfarlane, G.S.}, Katsikides, N., Granato, S. (2023). Travel Data Users Forum: Innovative Usage of GPS Trajectory Data: Present and Future. In \textit{Transportation Research Board Annual Meeting}. Invited panel discussion. Washington, D.C.
\item Antoniou, C., M\"ockel, R., \textbf{Macfarlane, G.S.}, Kotsopoulous, H., Llorca-Garcia, C., Erhardt, G.D., Mahajan, V.,  Schm\"ocker, J.D., \& Pereira, F. (2022). Transport Modeling using Publicly Available Data. Invited participants to a workshop hosted by Technische Universit\"at M\"unchen and the German Research Foundation.
\item \textbf{Macfarlane, G.S.}, \& Lant, N.J.$^\dagger$ (2022). How far are we from transportation equity? Measuring the effect of wheelchair use on daily activity patterns. In \textit{mobil.TUM 2022 – 12th International Scientific Conference on Mobility and Transport}. Lectern presentation. Singapore.
\item \textbf{Macfarlane, G.S.}, \& Atchley, S.H.$^*$, Day, C.S.$^\dagger$, Erhardt, G., \& Needell, Z. (2022). Simulating and prioritizing service areas for regionally exclusive microtransit operations. In \textit{mobil.TUM 2022 – 12th International Scientific Conference on Mobility and Transport}. Lectern presentation. Singapore.
 \ifdetail } \fi
\item \textbf{Macfarlane, G.S.}, Stucki,  E.$^\dagger$,  \& Copley, M.$^*$. (2021). Utility-Based Accessibility to Community Resources: An Application of Location-Based Services Data. In \textit{North American Regional Science Conference}. Denver, Colorado.
\item \textbf{Macfarlane, G.S.}, \& Kressner, J.D. (2018). Comparing the Daily Schedules in the NHTS from 2009 and 2017. In \textit{ National Household Travel Survey (NHTS) Data for Transportation Applications Workshop}. Poster. Washington, D.C.
\item \textbf{Macfarlane, G.S.}, \& Kressner, J.D. (2017). Modeling automated vehicles with a passive data model. In \textit{ Transportation Planning Applications Conference}. Poster. Raleigh, North Carolina.
\item Kressner, J.D., \textbf{Macfarlane, G.S.}, Donnelly, R., \& Huntsinger, L.F. (2016). Using passive data to build an agile tour-based model: A case study in Asheville. In \textit{ Innovations in Travel Modeling Conference}. Lectern presentation. Denver, Colorado. \ifdetail Citations: 7  \fi
\end{myenum}



%% ============================================================================
\clearpage
\section{\secfont Awards and\\ Honors}
\begin{description}
  \ifdetail {\color{NavyBlue} \fi
\item[\acc Most Influential Faculty] Given to the faculty member in the Civil Engineering program
whom graduating seniors name as the most influential on their undergraduate
education. Awarded by 2022 graduating class.
\item{\acc TUM Global Visiting Professor} Selected to enrich the vibrant research culture at 
the Technische Universit\"at M\"unchen by virtue of innovative approaches and to explore new, cutting-edge research fields. Awarded in 2023.
\item[\acc ASCE ExCEEd Teaching Fellow] Participated in week-long intensive teacher
development program. Class of 2022.
\ifdetail } \fi
\item[\acc Dwight David Eisenhower Graduate Fellowship] Full doctoral funding from
the Federal Highways Administration 2011-2013, one of five awards nationally.
Awarded supplemental grant in 2013.
\item[\acc Eno Center for Transportation Leadership Development Conference]
Participated in the 2012 program; nominated by the Ivan Allen, Jr.\ College of
Liberal Arts at Georgia Tech.
%\item[\acc Parsons Brinckerhoff - Jim Lammie Engineering Scholarship] Awarded
%to the top engineer in the 2011 American Public Transportation Foundation (APTF)
%competition. Sponsored by Mike Allegra, general manager of the Utah Transit
%Authority. Renewed in 2012.
%% Lesser awards (comment out for space saving)
%\item[\acc Gordon W. Schultz Graduate Fellowship] Given to the Georgia Tech
%student in travel demand modeling who exhibits innovation, problem-solving, and
%practical application.
%\item[\acc National Science Foundation Graduate Fellowship Program] Honorable
%Mention in 2011 and 2012, as a first- and second-year graduate student.
%\item[\acc Jim McGee Memorial Scholarship] Cash award from the Georgia chapter of the
%American Society of Highway Engineers, one of two awards in 2011.
%\item[\acc Georgia Department of Transportation Scholarship] One of ten cash awards in
%2010 to students from the southeastern United States.
%\item[\acc Office of Research and Creative Activities (ORCA) Grant] Competitive
%research grant to survey Chinese transportation planning practices, one of
%several undergraduate research awards from Brigham Young University.
%\item[\acc Freeman-Asia Award] Grant to study Chinese finance and globalized
%engineering at Nanjing University in the People's Republic of China from the
%Institute for International Education.
\end{description}

%% ============================================================================
\noindent\makebox[\linewidth]{\rule{\linewidth}{0.4pt}}
\section{\secfont External \\ Citizenship}



Panel Member, NCHRP 08-184, \textit{Framework for Assessing Induced Demand Effects of Various Roadway Investments}. (2023 - )

Member, Provo City Transportation and Mobility Advisory Commission

Transportation Research Board of the National Academies of Science:

\begin{itemize}
  \item AEP50: Travel Demand Forecasting Member of the committee (2019 --- ) on
  travel demand forecasting. Chair of the travel forecasting resources
  subcommittee and editor of \url{tfresource.org}.
  \item AMS20: Economics and Land Development Member of the committee (2014
  --- 2022). formerly standing committee on transportation and land use.
  \item Young Members Council (2019 --- 2021). Planning and Environment subcommittee
  chair.
\end{itemize}



Member of the following professional organizations:

\begin{itemize}
  \item American Society of Civil Engineers (2022 - )
  \item Zephyr Foundation (2020 - 2022).
  \item Institute of Transportation Engineers (2009-2013, 2018-2020)
  \item Tau Beta Pi (Utah $\beta$ '09).
\end{itemize}


\noindent\makebox[\linewidth]{\rule{\linewidth}{0.4pt}}
\section{\secfont Internal \\ Citizenship}
\begin{description}
  \item[Department honors coordinator] (2019 --- ). Encourage students to participate
  in the honors program, and participate on honors thesis committees in the
  department.
  \item[Department undergraduate committee] Chair (2023 --- ), Member (2021 --- ). Leading curriculum revisions for civil engineering program.
  \item[Department faculty development and capital improvement committee] (2018 --- 2021).
\end{description}


%% ==========================================

% \noindent\makebox[\linewidth]{\rule{\linewidth}{0.4pt}}
% \section{\secfont Skills}
% Significant computer software and programming ability, including expert skills
% in R, \LaTeX, and git. Also experienced with Java, C, Python, QGIS,
% Cube/Voyager,  Matlab, PTV Vissim / Visum, TransCAD (with GISDK).


%\clearpage
%\thispagestyle{empty}
%\section{\secfont References}
%\textit{ Rick Donnelly, Ph.D., AICP} \\
%6100 Uptown Boulevard NE, Suite 700, Albuquerque, NM 87110 \\
%DonnellyR@pbworld.com 1.505.878.6524
%
%\textit{ Leta F. Huntsinger, Ph.D., PE} \\
%434 Fayetteville Drive, Suite 1500, Raleigh NC 27601 \\
%Huntsinger@pbworld.com 1.919.836.4086
%
%%\textit{ Laurie A. Garrow, Ph.D.} \\
%%790 Atlantic Drive, Atlanta GA 30332-0355\\
%%laurie.garrow@ce.gatech.edu 1.404.385.6634
%
%\textit{ Juan Moreno-Cruz, Ph.D.} \\
%221 Bobby Dodd Way, Atlanta GA 30332\\
%juan.moreno-cruz@econ.gatech.edu 1.404.385.1100
%
%\textit{ Kari E. Watkins, Ph.D., P.E.} \\
%790 Atlantic Drive, Atlanta GA 30332-0355\\
%kari.watkins@ce.gatech.edu 1.206.250.4415
%
%%\textit{ Patricia L. Mokhtarian, Ph.D.} \\
%%790 Atlantic Drive, Atlanta GA 30332-0355\\
%%patmokh@ce.gatech.edu 1.404.385.1443
%
%%\textit{ Patrick S. McCarthy, Ph.D.} \\
%%221 Bobby Dodd Way, Atlanta GA 30332\\
%%mccarthy@gatech.edu 1.404.894.4914
%
%%\textit{ Mitsuru Saito, Ph.D., P.E.} \\
%368J Clyde Building, Provo UT 84602\\
%msaito@byu.edu 1.801.422.6326
\end{resume}
%\vfill\moveleft.25\hoffset\centerline{References are available upon request.}



\end{document}
