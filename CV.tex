\documentclass[margin,line]{res}

%This lets me make hyper references to BYU and Georgia Tech
\usepackage{etoolbox}
\usepackage{booktabs}
\usepackage{enumitem}
\usepackage{etaremune}
\usepackage[colorlinks=false]{hyperref}


% page numbers
\usepackage{fancyhdr}
\pagestyle{fancy}
\renewcommand{\headrulewidth}{0pt}
\renewcommand{\footrulewidth}{0pt}

\oddsidemargin -.5in
\evensidemargin -.5in
\textwidth=6.0in
\itemsep=0in
\parsep=0in

% if detailtrue, then the CV will show notes on publication venue
% as well as student evaluations for each course.
\newif\ifdetail
\ifcsdef{myvar}{\myvar}{}

%\detailtrue

%% ==========================================
%This allows me to use Bibtex to create my publication list
%\usepackage{natbib}
%\makeatletter
%\renewenvironment{thebibliography}[1]
%     {%\section*{\bibname}% <-- remove this to make section naming correct
%      \@mkboth{\MakeUppercase\bibname}{\MakeUppercase\bibname}%
%      \list{\@biblabel{\@arabic\c@enumiv}}%
%           {\settowidth\labelwidth{\@biblabel{#1}}%
%            \leftmargin\labelwidth
%            \advance\leftmargin\labelsep
%            \@openbib@code
%            \usecounter{enumiv}%
%            \let\p@enumiv\@empty
%            \renewcommand\theenumiv{\@arabic\c@enumiv}}%
%      \sloppy
%      \clubpenalty4000
%      \@clubpenalty \clubpenalty
%      \widowpenalty4000%
%      \sfcode`\.\@m}
%     {\def\@noitemerr
%       {\@latex@warning{Empty `thebibliography' environment}}%
%      \endlist}
%\makeatother

%\usepackage[resetlabels]{multibib}
%  \newcites{j,c,w}{{Test},{Test1},{Test2}}
% j- journals c-conferences p-presentations
\setdescription{font=\normalfont}

%% ==========================================
% This defines the list types and formatting
\newenvironment{list1}{
  \begin{list}{\ding{113}}{%
      \setlength{\itemsep}{0in}
      \setlength{\parsep}{0in} \setlength{\parskip}{0in}
      \setlength{\topsep}{0in} \setlength{\partopsep}{0in}
      \setlength{\leftmargin}{0.17in}}}{\end{list}}
\newenvironment{list2}{
  \begin{list}{$\bullet$}{%
      \setlength{\itemsep}{0in}
      \setlength{\parsep}{0in} \setlength{\parskip}{0in}
      \setlength{\topsep}{0in} \setlength{\partopsep}{0in}
      \setlength{\leftmargin}{0.2in}}}{\end{list}}

\newcounter{enuminitialize}

\newenvironment{myenum}[1][]
{%
 \setcounter{enuminitialize}{#1}
 \addtocounter{enuminitialize}{2}
 \begin{enumerate}[left= 4pt, itemsep=8pt, start=\value{enuminitialize}, label=\arabic*\addtocounter{enumi}{-2}]
}
{%
 \end{enumerate}
}


% define fonts
%\usepackage{fontspec}
%\setmainfont[Mapping=tex-text]{Palatino}
%\newfontface\acc{Marcellus SC}
\newcommand{\secfont}{\scshape }
\newcommand{\acc}{\scshape }

\begin{document}
% Center the name over the entire width of resume:
 \moveleft.5\hoffset\centerline{\LARGE\scshape Gregory S.  Macfarlane}
\vspace{.05in}
 \moveleft.5\hoffset\centerline{Brigham Young University}
 \moveleft.5\hoffset\centerline{
	 \href{mailto:gregmacfarlane@byu.edu}{gregmacfarlane@byu.edu}
   801.422.8505}
\vspace{.05in}
% address begins here
% Again, the address lines must be centered over entire width of resume:
 \moveleft.5\hoffset\centerline{430 Engineering Building}
 \moveleft.5\hoffset\centerline{Provo, UT 84602}

\begin{resume}

%\section{\sc Contact\\ Information}
\vspace{.05in}



%% ============================================================================
\section{\secfont Education}
\href{http://www.gatech.edu}{\acc Georgia Institute of Technology}
\\
	\vspace*{-.1in}
	\begin{list1}
	\item[] Ph.D., Transportation Systems Engineering  \hfill{May 2014}
%	\begin{list2}
%  	\item[] Advisor: Laurie A. Garrow
%		\item[] Dissertation: ``Using Big Data to Model Travel Behavior: Applications to Vehicle Ownership and Willingness-to-Pay for Transit Accessibility''
%%		\item[] GPA: 3.86/4.0
%%		\item[] Committee: Laurie A. Garrow (Chair - CEE), Juan Moreno-Cruz
%%(Economics), \\ Patricia L. Mokhtarian (CEE), Kari E. Watkins (CEE), Patrick S.
%%McCarthy (Economics), Jeffrey P. Newman (CEE)
%  \end{list2}
	\vspace*{.05in}
	\item[]M.S., Economics \hfill{May 2014}
	\end{list1}

\href{http://www.byu.edu}{\acc Brigham Young University}\hfill{December 2009}
\\
\vspace*{-.1in}
\begin{list1}
  \item[] B.S. with University Honors, Civil Engineering
	\item[] Minor	degrees in Mathematics and Asian Studies
%  \begin{list2}
%		\item[] Honors Thesis: ``Delay Patterns
%			and Perceptions at Free Right-turn Channelized Intersections''
%		\item[] Advisor: Mitsuru Saito
%%		\item[] GPA: 3.7/4.0
%    \item[] Studies Abroad: Nanjing, China (Engineering);
%			Naples, Italy (Arch{\ae}ology)
%	\end{list2}
\end{list1}

\noindent\makebox[\linewidth]{\rule{\linewidth}{0.4pt}}
%\textit{ Research Interests ---} Passive data and its applications in transport and
%land use modeling and forecasting.
%% ============================================================================
\section{\secfont Academic Experience}

{\acc Brigham Young University}

\vspace{-.4cm}
\textit{Assistant Professor} \hfill {November 2018 --- }\\

\vspace{-.4cm}
{\acc University of North Carolina, Chapel Hill}

\vspace{-.4cm}
\textit{Adjunct Lecturer/Teaching Assistant} \hfill {January 2017 --- May 2017}\\

\vspace{-.4cm}
{\acc Georgia Institute of Technology}

\vspace{-.4cm}
\textit{Post-doctoral Researcher} \hfill {January 2014 --- May 2014}\\

\vspace{-.4cm}
\section{\secfont Research\\ Interests}
Transportation planning and engineering, travel demand modeling,
passive transportation data, spatial and social correlation.

\noindent\makebox[\linewidth]{\rule{\linewidth}{0.4pt}}

%% ============================================================================
%\noindent\makebox[\linewidth]{\rule{\linewidth}{0.4pt}}
\section{\secfont Professional \\ Experience}

Registered professional engineer in North Carolina, license \#044518

{\acc Transport Foundry} Atlanta, Georgia

\vspace{-.3cm}
\textit{Transportation Engineer} \hfill {April 2017 --- October 2018}\\
Developed a data-driven travel demand model from passive data sources.

{\acc WSP | Parsons Brinckerhoff} Raleigh, North Carolina

\vspace{-.3cm}
\textit{Technical Principal, Systems Analysis Group} \hfill {June 2014 --- April 2017}\\
Developed advanced travel demand models for public sector clients.



{\acc Utah Transit Authority} Salt Lake City, Utah

\vspace{-.3cm}
\textit{Strategic Planning Intern} \hfill {May 2009 --- June 2010}\\
Developed transit operating scenarios for the Wasatch Front long-range
transportation plan and for UTA's internal scenario planning and programming purposes.

{\acc Hales Engineering} Lehi, Utah

\vspace{-.3cm}
\textit{Engineering Intern} \hfill {July 2008 --- May 2009}\\
Prepared traffic impact analyses for commercial and residential developments.


%% ==========================================
% BibTex reference management

% Journal Articles
%\section{\secfont Under \\ Review}
%\bibentry{epb_parkaccess}

%\noindent\makebox[\linewidth]{\rule{\linewidth}{0.4pt}}

\section{\secfont Refereed \\ Journal\\ Articles}
First author or first faculty author on 13 of 27 total journal articles.
$^\dagger$indicates BYU graduate student authors, $^*$indicates BYU undergraduate authors.
\ifdetail 
Paper 1 came from my undergraduate work, papers 2 through 6 were from my doctoral research, and
papers 7 onward represent work completed during my time on the faculty at BYU. Number of citations
are from Google Scholar as of December 2021. \fi
\vspace{.3cm}

\begin{myenum}[27]


  \item Erhardt, G.D., Guan, H.Z. Lee, D., \textbf{Macfarlane, G.S.}, Van Hentenryck,
  P. (2024). Comprehensive studies on on-demand multimodal transit systems with
  two case studies in San Francisco and Salt Lake City. Working paper.

  \item Tenenboim, E., Xu, Y., Erhardt, G.D.,  \textbf{Macfarlane, G.S.}, \& Peeta, S.
  (2024).  A model of ride-hailing driver participation: shift duration, start
  time, and start location. Working paper.

  \item Day, C.$^\dagger$, \textbf{Macfarlane, G.S.}, Atchley, S.H. $^*$, Erhardt, G.D., Ven Hentenryck, P., Watkins, K.E. (2024). 
  Implementation and quantitative evaluation of multi-modal optimization and simulation for transit and ridehail competitive analysis. 
  Working paper.

  \item \textbf{Macfarlane, G.S.} \& Gray, N.M.$^\dagger$ (2024). 
  Evaluating the impacts of parameter uncertainty in a practical transportation demand model. Working paper.

  \item Atchley, S.H.$^\dagger$, Mansfield, K.A.$^*$, \textbf{Macfarlane, G.S.}
  (2024). A comparative illustration of trip- and activity-based modeling
  methods. Under review at \textit{Transportation Research Record}.

  \item Hyer, J.C.$^\dagger$, Schultz, G.G., \textbf{Macfarlane, G.S.} An analysis
  of UDOT’s expanded incident management team program. Under review at
  \textit{Transportation Research Record}.

  \item \textbf{Macfarlane, G.S.}, Stucki, E.$^\dagger$, Salmon, M.$^*$, Redelfs, A.H.,
  Spruance, L.A. (2024). Where’s dinner coming from? A utility-based
  investigation of access to nutrition in Utah. Under review at \textit{Journal of Transport and Land Use}.

  \item Sivakumar, A., Jones, P. M\"ockel, R., Moreno Chou, A.T., Erhardt, G.,
  \textbf{Macfarlane, G.S.}. (2024). The ‘activity-based’ approach: a new
  perspective for addressing the major environmental and resource challenges
  faced by societies. Under review at \textit{Proceedings of the National Academy of Science}. 

  \item \textbf{Macfarlane, G.S.}, Barnes, M.$^\dagger$, \& Gray, N.M.$^*$ (2024). 
  A utility-based approach to modeling systemic resilience of highway networks with an application in Utah.
  \textit{Journal of Transportation Engineering Part A: Systems}, forthcoming.

  \item Wang, B.$^\dagger$, Fulda, N., Huang, Z.Y.$^*$, Schultz, G.G., \textbf{Macfarlane, G.S.}, Arnesen, J.$^*$, Khayyat, A.$^*$ (2024). 
  Predicting directional traffic volume at intersections with automated traffic signal performance measures data using machine learning algorithms.
 \textit{Transportation Research Record}. \url{https://doi.org/10.1177/03611981241252829}

  \item \textbf{Macfarlane, G.S.}, Riches, G.$^\dagger$, Youngs, E.K.$^\dagger$, Nielsen, J.A. (2024). Classifying location points as daily activities using simultaneously optimized DBSCAN-TE parameters. 
  \textit{Findings}. \url{https://doi.org/10.32866/001c.116197}

  \item Turley Voulgaris, C., \textbf{Macfarlane, G.S.}, Kaylor, J. (2024).
  Whose emissions are these anyway? Estimating vehicle miles traveled to account for site-level climate impacts.
  \textit{Journal of the American Planning Association}.  \url{https://doi.org/10.1080/01944363.2023.2298962}

  \item  Wang, B.$^\dagger$, Schultz, G.G., \textbf{Macfarlane, G.S.}, Eggett, D.L., 
  \& Davis, M.C.$^*$ (2023) A methodology to detect traffic data anomalies in automated traffic 
  signal performance measures. \textit{Future Transportation}. 3(4), 1175-1194. \url{https://doi.org/10.3390/futuretransp3040064}

  \item Daines, T.J.$^\dagger$, Schultz, G.G., \textbf{Macfarlane, G.S.}, \&
  Ward, C.$^*$ (2022). Evaluating real time ramp meter queue length estimation.
  \textit{Future Transportation}, 2(4), 807-827. \url{https://doi.org/10.3390/futuretransp2040045}

  \item \textbf{Macfarlane, G.S.}, Stucki, E.$^\dagger$, Redelfs, A.H., \& Spruance, L.A. (2022).
  Beyond proximity: utility-based access from location-based services data.
  \textit{International Journal of Environmental Research and Public Health}, 19(19), 12352.
  \url{https://doi.org/10.3390/ijerph191912352}.

  \item \textbf{Macfarlane, G.S.}, Turley Voulgaris, C., \& Tapia, T. (2022).
  City parks and slow streets: a utility-based access and equity analysis.
  \textit{Journal of Transport and Land Use}. 15(1): 587-612.
  \url{https://doi.org/10.5198/jtlu.2022.2009}

  \item Wang, B.$^\dagger$, Schultz, G.G., \textbf{Macfarlane, G.S.}, \& McCuen,
  S.$^*$ (2022). Evaluating signal systems using automated traffic signal
  performance measures. \textit{Future Transportation}. 2(3):  659-674.
  \url{https://doi.org/10.3390/futuretransp2030036}.

  \item \textbf{Macfarlane, G.S.}, Sheffield, M.H.$^\dagger$, Bennet, L.S.$^\dagger$, \& Schultz, G.G. (2021).
  The effect of transit signal priority on bus rapid transit headway adherence.
  \textit{Findings}, June. \url{https://doi.org/10.32866/001c.24499}.

  \item\textbf{Macfarlane, G.S.}, Hunter, C.$^*$, Martinez, A.$^*$, \& Smith, E.$^*$  (2021).
  Rider perceptions of an on-demand microtransit service in Salt Lake County, Utah
  \textit{ Smart Cities} 4(2): 717-727.
  \url{https://doi.org/10.3390/smartcities4020036} \ifdetail Citations: 1 \fi

  \item\textbf{Macfarlane, G.S.}, Boyd, N., Taylor, J.E., \& Watkins, K. (2021) Modeling the impacts of park access
on health outcomes: A utility-based accessibility approach. \textit{ Environment and
Planning B: Urban Analytics and City Science}, 48(8), 2289–2306. \url{https://doi.org/10.1177/2399808320974027}

  \item Glenn, J., Bluth, M.$^*$, Christianson, M.$^*$, Pressley, J.$^*$, Taylor, A., \textbf{Macfarlane, G.S.}, \& Chaney, R. A. (2020).
  Considering the potential health impacts of electric scooters: an analysis of user reported behaviors in Provo, Utah.
  \textit{ International Journal of Environmental Research and Public Health}, 17(17), 6344. \url{https://doi.org/10.3390/ijerph17176344} \ifdetail Citations: 5  \fi

  \item\textbf{Macfarlane, G.S.}, Garrow, L.A., \& Moreno-Cruz, J. (2015). Do Atlanta
residents value MARTA? Selecting an autoregressive model to recover willingness
to pay. \textit{ Transportation Research Part A: Policy and Practice}, 78, 214–230.
\url{https://doi.org/10.1016/j.tra.2015.05.010} \ifdetail Citations: 8 \fi

  \item\textbf{Macfarlane, G.S.}, Garrow, L.A., \& Mokhtarian, P. L. (2015). The influences of
past and present residential locations on vehicle ownership decisions.
\textit{ Transportation Research Part A: Policy and Practice}, 74, 186–200.
\url{https://doi.org/10.1016/j.tra.2015.01.005} \ifdetail Citations: 40  \fi

  \item Brakewood, C., \textbf{Macfarlane, G.S.}, \& Watkins, K.E. (2015). The impact of
real-time information on bus ridership in New York City. \textit{ Transportation Research
Part C: Emerging Technologies}, 53, 59–75. \url{https://doi.org/10.1016/j.trc.2015.01.021} \ifdetail Citations: 144  \fi

  \item Binder, S., \textbf{Macfarlane, G.S.}, Garrow, L.A., \& Bierlaire, M. (2014).
Associations among household characteristics, vehicle characteristics and
emissions failures: An application of targeted marketing data. \textit{ Transportation
Research Part A: Policy and Practice}, 59, 122–133.
\url{https://doi.org/10.1016/j.tra.2013.11.005}\ifdetail Citations: 16  \fi

  \item Wall, T.A., \textbf{Macfarlane, G.S.}, \& Watkins, K.E. (2014). Exploring the use of
egocentric online social network data to characterize individual air travel
behavior. \textit{ Transportation Research Record}, 2400, 78–86.
\url{https://doi.org/10.3141/2400-09} \ifdetail Citations: 9  \fi

  \item McBride, J.H., Keach, R. W., Macfarlane, R.T., De Simone, G.F., Scarpati, C.,
Johnson, D.J., \textbf{Macfarlane, G.S.}, \& Weight, R.W.R. (2009). Subsurface visualization using
ground-penetrating radar for archaeological site preparation on the northern
slope of Somma-Vesuvius: a Roman site, Pollena-Trocchia, Italy. \textit{ Il Quaternario,
Italian Journal of Quaternary Sciences}, 22(1), 39–52. \url{https://portal.issn.org/resource/ISSN/0394-3356} \ifdetail Citations: 5  \fi
\end{myenum}

\ifdetail
\section{\secfont{Venue Notes}}
\begin{description}
  \item[\textit{Journal of Transport and Land Use}] is is the leading
  international journal that publishes original interdisciplinary papers on the
  interaction of transport and land use. CiteScore: 4.3; 33/241 in urban studies.
  5-year impact factor: $2.713$.  Publisher: University of Minnesota.
  \item[\textit{Transportation Research Part A: Policy and Practice}] is
   a leading international journal with robust peer review focusing on
   transportation policy  analysis and the planning of transportation systems.
   CiteScore: 8.5; 17/318 in civil engineering. Impact factor: $5.594$. Publisher: Elsevier.
  \item[\textit{Environment and Planning B: Urban Analytics and City Science}]
  is a leading international journal with robust peer review publishing
  cutting-edge research in analytical methods for urban planning and
  design. CiteScore 4.6; 6/138 in architecture. 5-year impact factor: $3.889$. Publisher: Sage.
  \item[\textit{Transportation Research Part C: Emerging Technologies}] is a
  leading international journal with robust peer review focusing on applications
  and implications of technology in transportation systems.
  CiteScore: 14.0; 3/318 in civil engineering. 5-year impact factor: $8.089$. Publisher: Elsevier.
  \item[\textit{Findings}] is an interdisciplinary, independent, community-led,
  peer-reviewed, open access journal focused on short, clear, and pointed
  research results. The journal was established in 2019. Publisher: University of
  Sydney and McGill University.
  \item[\textit{International Journal of Environmental Research and Public Health}]
  is an interdisciplinary, peer-reviewed, open access journal. CiteScore: 3.4;
  179/526 in public health.  5-year impact factor: $2.789$. Publisher: MDPI.
  \item[\textit{Smart Cities}] is an international, scientific, peer-reviewed,
  open access journal on the science and technology of smart cities. This
  journal was established in 2019. Publisher: MDPI.
\end{description}
\fi

% \textbf{Macfarlane, G.S.}, \& Lant, N.$^\dagger$ (2021). Identifying Segmentation Strategies in a Daily Activity Pattern Model for Wheelchair Users.

\noindent\makebox[\linewidth]{\rule{\linewidth}{0.4pt}}
\section{\secfont Peer- Reviewed Conference \\ Papers}
All listed conference publications are full papers and include at least a
single-blind review process with multiple expert reviewers for consideration.
\ifdetail Papers 1 and 2 resulted from my undergraduate honors thesis, papers 3 through 5
resulted from my doctoral work, and items since item 6 represent work completed since joining the faculty at BYU.\fi
\vspace{.3cm}
\begin{myenum}[11]
%\item \textbf{Macfarlane, G.S.}, Stucki, E.$^\dagger$, Salmon, M.$^*$, Redelfs, A.H., Spruance, L.A. (2024). Where's dinner coming from? A utility-based investigation of access to nutrition in Utah. In \textit{World Symposium of Transport and Land Use Research}. Bogot\'a, Colombia.


\item Jarvis, D.L.$^\dagger$, \textbf{Macfarlane, G.S.}, Woolley, B.$^*$, Schultz, G.G. (2024). Simulating incident management team response and performance. \textit{Procedia Computer Science}. 238,
pp. 91-96.  \url{https://doi.org/10.1016/j.procs.2024.06.002}

\item Apelu, D.$^*$, \textbf{Macfarlane, G.S.}, Guthrie, W.S., Adams, N.$^*$, Mazzeo, B. (2023). Measuring Pavement Smoothness From the Perspective of E-Scooters. In \textit{2023 Intermountain Engineering, Technology, and Computing Conference (i-ETC)}. IEEE XPlore, \url{https://doi.org/10.1109/IETC57902.2023.10152077}

\item \textbf{Macfarlane, G.S.}, Lant, N.$\dagger$ (2023). How Far Are We From Transportation Equity? Measuring the Effect of Wheelchair Use on Daily Activity Patterns. In: Antoniou, C., Busch, F., Rau, A., Hariharan, M. (eds) Proceedings of the 12th International Scientific Conference on Mobility and Transport. \textit{Lecture Notes in Mobility.} Springer, Singapore. \url{https://doi.org/10.1007/978-981-19-8361-0_10}

\item Turley Voulgaris, C., \textbf{Macfarlane, G.S.}, Kaylor, J., Su, T., Bauranov, A. (2022). Whose emissions are these anyway? Estimating vehicle miles traveled to account for site-level climate impacts. In \textit{Transportation Research Board Annual Meeting}. Washington, D.C.

\item \textbf{Macfarlane, G.S.}, Stucki,  E.$^\dagger$,  \& Copley, M.$^*$. (2021). Utility-Based Accessibility to Community Resources: An Application of Location-Based Services Data. In \textit{North American Regional Science Conference}. Denver, Colorado.

\item \textbf{Macfarlane, G.S.}, \& Tapia, T. (2020). Developing a Park Activity Location Choice Model from Passive Origin-Destination Data Tables. In \textit{ Transportation Research Board Annual Meeting}. Washington, D.C.

\item \textbf{Macfarlane, G.S.}, \& Moreno-Cruz, J. (2015). The Association Between Public Transportation Infrastructure and Home Price Growth and Stability. \textit{North American Regional Science Conference}. Portland, Oregon.

\item Zhang, B., \textbf{Macfarlane, G.S.}, Wall, T.A., \& Watkins, K.E. (2014). Friendship Influences on Air Travel: A Social Autoregressive Analysis. In \textit{ North American Regional Science Conference}. Washington, D.C.

\item \textbf{Macfarlane, G.S.}, Moreno-Cruz, J., \& Garrow, L. A. (2013). Does Atlanta value MARTA? Selecting an autoregressive model to recover willingness-to-pay. In \textit{ North American Regional Science Conference}. Atlanta, Georgia.

\item \textbf{Macfarlane, G.S.}, Saito, M., \& Schultz, G.G. (2011). Delay underestimation at free right-turn channelized intersections. In \textit{ 6th International Symposium on Highway Capacity and Quality of Service} (Vol. 16, pp. 560–567). https://doi.org/10.1016/j.sbspro.2011.04.476 \ifdetail Citations: 6  \fi

\item \textbf{Macfarlane, G.S.}, Saito, M., \& Schultz, G.G. (2011). Driver perceptions at free right-turn channelized intersections. In \textit{ T\&DI Congress 2011: Integrated Transportation and Development for a Better Tomorrow} (Vol. 398, pp. 108–108). ASCE. https://doi.org/10.1061/41167(398)108 \ifdetail Citations: 3  \fi
\end{myenum}


\noindent\makebox[\linewidth]{\rule{\linewidth}{0.4pt}}
\section{\secfont Reports}
\ifdetail These are technical reports completed under contract for the sponsoring agency;
each report was reviewed by a technical advisory committee prior to publication. Item 1 
resulted from postdoctoral activities at Georgia Tech, 
items 2 and 3 from my consulting practice, and items 4 through the present from my work 
since joining BYU.\fi
\vspace{0.3cm}
\begin{myenum}[12]
\item \textbf{Macfarlane, G.S.}, Atchley, S.H.$^\dagger$, Mansfield, K.A.$^*$, Baird, T. \& Gresham, C. (2024). 
  \textit{Activity-based Model Implementation and Analysis Considerations. }  (No. UT-24.16).  Utah Dept.\ of Transportation. Division of Research. 

\item \textbf{Macfarlane, G.S.},  Jarvis, D.L.$^\dagger$, Woolley, B.$^*$, \& Schutz, G.G. (2024).
  \textit{Simulating Incident Management Team Response and Performance.} (No. UT-23.22). 
  Utah Dept.\ of Transportation. Division of Research. \url{https://rosap.ntl.bts.gov/view/dot/74034}

\item Schultz, G.G., Hyer, J.$^\dagger$, Holdsworth, H.$^*$, Eggett, D.L., \& \textbf{Macfarlane, G.S.} (2023).
  \textit{Analysis of Benefits of UDOT’s Expanded Incident Management Team Program.} (No. UT-23.05). 
  Utah Dept.\ of Transportation. Division of Research. \url{https://doi.org/10.21949/1528563} 

  \item  \textbf{Macfarlane, G.S.}, Atchley, S.H.$^\dagger$ (2023).
  \textit{Identifying Microtransit Service Areas through Microsimulation}. 
  (No. UT-23.01). Utah Dept.\ of Transportation. Division of Research.
  \url{https://rosap.ntl.bts.gov/view/dot/66312}

\item Schultz, G.G. \textbf{Macfarlane, G.S.}, Wang, B.$^\dagger$, \& Davis,
  M.C.$^*$ (2022). \textit{Detecting Traffic Data Anomalies in Longitudinal Signal
  Performance Measures}. (No. UT-22.21). Utah Dept.\ of Transportation. Division of
  Research. \url{https://rosap.ntl.bts.gov/view/dot/65833}

\item Schultz, G. G., \textbf{Macfarlane, G.S.}, Daines, T.J.$^\dagger$, Ward,
  C.K.$^*$, Umphress, J.$^*$ (2022). \textit{Evaluating Ramp Meter Wait Time in
  Utah}. (No. UT-21.06). Utah Dept.\ of Transportation. Division of Research.
  \url{https://rosap.ntl.bts.gov/view/dot/61507}

\item \textbf{Macfarlane, G.S.}, Lant, N.J.$^\dagger$, (2021).
  \textit{Estimation and Simulation of Daily Activity Patterns for Individuals
  Using Wheelchairs} (No. UT-21.10). Utah Dept.\ of Transportation. Division of
  Research. \url{https://rosap.ntl.bts.gov/view/dot/54639/dot_54639_DS1.pdf}

\item Schultz, G. G., \textbf{Macfarlane, G.S.}, Wang, B.$^\dagger$, \& McCuen,
  S.$^*$ (2020). \textit{Evaluating the Quality of Signal Operations Using Signal
  Performance Measures} (No. UT-20.08). Utah Dept.\ of Transportation. Division of
  Research. \url{https://rosap.ntl.bts.gov/view/dot/54639/dot_54639_DS1.pdf}

\item \textbf{Macfarlane, G.S.} \& Copley, M.J.$^*$ (2020). \textit{ A Synthesis
  of Passive Third-Party Data sets used for Transportation Planning.} (No.
  UT-20.20). Utah Dept.\ of Transportation. Division of Research.
  \url{https://rosap.ntl.bts.gov/view/dot/54890/dot_54890_DS1.pdf}

\item Zalewski, A., Sonenklar, D., Cohen, A., Kressner, J., \&
  \textbf{Macfarlane, G.S.} (2019). \textit{ Public Transit Rider
  Origin–Destination Survey Methods and Technologies}. TCRP Synthesis of Transit
  Practice 138. Transportation Research Board.
  \url{http://www.trb.org/Main/Blurbs/179008.aspx} \ifdetail Citations: 1  \fi

\item Miller, H., O'Kelly, M., Jaegal, Y., Bachman, W., Huntsinger, L., \&
  \textbf{Macfarlane, G.S.} (2017). \textit{Estimating External Travel Using
  Purchased Third-Party Data.} Research Report 134877, the Ohio Department of
  Transportation, Office of Statewide Planning \& Research. \ifdetail Citations: 1 \fi

\item Cruz, J., \textbf{Macfarlane, G.S.}, Xu, Y., Rodgers, M.O., \& Guensler,
  R. (2015). \textit{Sustainable Transportation Curricula}. National Center for
  Sustainable Transportation. \url{https://escholarship.org/uc/item/3c13q43c}.
\end{myenum}

\noindent\makebox[\linewidth]{\rule{\linewidth}{0.4pt}}
\section{\secfont Presentations}
\ifdetail This includes invited presentations to academic and non-academic audiences, as
well as presentations resulting from abstract-only submission. Includes both
lectern sessions and posters. Item 1 came from my undergraduate honors thesis, items 2 through 4 from doctoral research,
items 5 through 10 from my work as a consultant, and items 11 through the present represent work
completed during my time at BYU.\fi

\vspace{0.3cm}



\begin{myenum}[33]
\item \textbf{Macfarlane, G.S.}, Baird, T., \& Atchley, S.H.$^\dagger$ (2024). Activity-based models in Utah: A comparative illustration. In \textit{Utah Model User's Group Meeting}. Working group presentation. Orem, Utah. 
\item Turley Voulgaris, C., Jensen, A.F., \textbf{Macfarlane, G.S.} (2024). Children’s mode choice and independence for the journey to school. In \textit{17th International Conference on Travel Behavior Research}. Lectern presentation. Vienna, Austria.
\item Baird, T., Gresham, C., Atchley, S.H.$^\dagger$ \& \textbf{Macfarlane, G.S.} (2024). A Clearer Crystal Ball? Practitioner perspectives on improving travel models. In \textit{Mountain District ITE Meeting}. Lectern presentation. Big Sky, Montana. 
\item Jarvis, D.L.$^\dagger$, \textbf{Macfarlane, G.S.}, Woolley, B.$^*$, Schultz, G.G. (2024). Simulating incident management team response and performance. In \textit{The 15th International Conference on Ambient Systems, Networks and Technologies (ANT)}. Hasselt, Belgium. Lectern presentation.
\item \textbf{Macfarlane, G.S.}, Barnes, M.$^\dagger$, \& Gray, N.M.$^*$ (2024).  A utility-based approach to modeling systemic resilience of highway networks with an application in Utah.  In \textit{Transportation Research Board Annual Meeting}. Poster presentation. Washington, D.C.
\item Brown, N., \textbf{Macfarlane, G.S.}., Baird, T., \& Gresham, C. (2023). Activity-based models: A clearer crystal ball? in \textit{Utah Transportation Conference}. Lectern presentation. Sandy, Utah.
\item \textbf{Macfarlane, G.S.} Modeling with Big Data. (2023) At \textit{Technische Universit\"at M\"unchen}, invited lecture in Dr. Rolf M\"ockel's travel modeling course.
\item Guan, H.Z., Van Hentenryck, P., Erhardt, G.D., \textbf{Macfarlane, G.S.}, \& Watkins, K.E. (2023). Lessons from the design of on-demand multimodal transit systems in two cities. In \textit{Innovations in Transportation Analysis and Planning Conference}. Lectern presentation. Indianapolis, Indiana. \textit{Winner, best presentation by a student author.}
\item Ducuara, A. Holtrop-Kohl, L., Begay, S., Spruance, L., Redelfs, A., \textbf{Macfarlane, G.S.} (2023). Hungry for change: how cutting-edge research is helping to reduce food deserts in Utah. In \textit{Move Utah Summit}. Moderated panel discussion. Salt Lake City, Utah.
\item Singh, G., Young, S., \textbf{Macfarlane, G.S.}, Katsikides, N., Granato, S. (2023). Travel Data Users Forum: Innovative Usage of GPS Trajectory Data: Present and Future. In \textit{Transportation Research Board Annual Meeting}. Invited panel discussion. Washington, D.C.
\item \textbf{Macfarlane, G.S.}, Barnes, M.$^\dagger$, \& Gray, N.$^\dagger$. (2022). Evaluating systemic resiliency in Utah. In \textit{Utah Dept. of Transportation Annual Conference}. Lectern presentation. Sandy, Utah.
\item \textbf{Macfarlane, G.S.}, Day, C.S.$^\dagger$, \& Atchley, S.H.$^\dagger$. (2022). Modeling novel transport modes. In \textit{Utah Dept. of Transportation Annual Conference}. Lectern presentation. Sandy, Utah.
\item Antoniou, C., M\"ockel, R., \textbf{Macfarlane, G.S.}, Kotsopoulous, H., Llorca-Garcia, C., Erhardt, G.D., Mahajan, V.,  Schm\"ocker, J.D., \& Pereira, F. (2022). Transport Modeling using Publicly Available Data. Invited participants to a workshop hosted by Technische Universit\"at M\"unchen and the German Research Foundation.
\item \textbf{Macfarlane, G.S.}, \& Lant, N.J.$^\dagger$ (2022). How far are we from transportation equity? Measuring the effect of wheelchair use on daily activity patterns. In \textit{mobil.TUM 2022 – 12th International Scientific Conference on Mobility and Transport}. Lectern presentation. Singapore.
\item \textbf{Macfarlane, G.S.}, \& Atchley, S.H.$^*$, Day, C.S.$^\dagger$, Erhardt, G., \& Needell, Z. (2022). Simulating and prioritizing service areas for regionally exclusive microtransit operations. In \textit{mobil.TUM 2022 – 12th International Scientific Conference on Mobility and Transport}. Lectern presentation. Singapore.
\item Anderson, S.$^*$, \textbf{Macfarlane, G.S.}, \& Schultz, G.G. (2022). Developing a New Method to Analyze Speed and Braking Data Using V2X Technology. In \textit{Utah Conference of Undergraduate Research}. Poster. St.\ George, Utah.
\item \textbf{Macfarlane, G.S.} (2022). Using Big Data to Evaluate Equitable Access to Community Resources. In \textit{Transportation Research Board Annual Meeting}. Invited lectern presentation. Washington, D.C.
\item \textbf{Macfarlane, G.S.}, Stucki,  E.$^\dagger$,  \& Copley, M.$^*$. (2021). Utility-Based Accessibility to Community Resources: An Application of Location-Based Services Data. In \textit{North American Regional Science Conference}. Denver, Colorado.
\item \textbf{Macfarlane, G.S.}, Lant, N.J.$^\dagger$, (2021). Estimation and Simulation of Daily Activity Patterns for Individuals Using Wheelchairs. In \textit{Utah Dept. of Transportation Annual Conference.} Lectern presentation. Sandy, Utah.
\item \textbf{Macfarlane, G.S.} \& Copley, M.J.$^*$ (2020).  A Synthesis of Passive Third-Party Data sets used for Transportation Planning. In \textit{Utah Dept. of Transportation Annual Conference}. Poster. Sandy, Utah.
\item Turley Voulgaris, C., \textbf{Macfarlane, G.S.}, Kaylor, J., Su, T., Bauranov, A. (2021). Whose emissions are these anyway? Estimating vehicle miles traveled to account for site-level climate impacts. In \textit{Association of Collegiate Schools of Planning Annual Conference}. Lectern presentation. Miami, Florida.
\item \textbf{Macfarlane, G.S.}, Boyd, N., Taylor, J.E., \& Watkins, K.E. (2019). Modeling the impacts of park access on health outcomes: a choice-based accessibility approach. In \textit{ Greater and Greener 2019}. Workshop presentation. Denver, Colorado.
\item Bernardin, V., Gallup, A., Lee, B., Johnson, C., \textbf{Macfarlane, G.S.}, Elgar, I., Wertman, R. (2019). How to be a Good Big Data Consumer. In \textit{ Transportation Planning Applications Conference}. Panel discussion. Portland, Oregon.
\item \textbf{Macfarlane, G.S.}, \& Kressner, J.D. (2018). Comparing the Daily Schedules in the NHTS from 2009 and 2017. In \textit{ National Household Travel Survey (NHTS) Data for Transportation Applications Workshop}. Poster. Washington, D.C.
\item \textbf{Macfarlane, G.S.}, Bettinardi, A.O., \& Donnelly, R. (2017). SWIMR: Visualizing complex longitudinal indicators for a statewide integrated land use and transport model in Oregon. In \textit{Transportation Planning Applications Conference}. Lectern presentation. Raleigh, North Carolina.
\item Boyd, N., \textbf{Macfarlane, G.S.}, Watkins, K.E., \& Ederer, D. (2017). Accessibility to urban parks and health outcomes on the neighborhood level. In \textit{ American Public Health Association Annual Meeting}. Poster. Atlanta, Georgia.
\item \textbf{Macfarlane, G.S.}, \& Kressner, J.D. (2017). Modeling automated vehicles with a passive data model. In \textit{ Transportation Planning Applications Conference}. Poster. Raleigh, North Carolina.
\item Kressner, J.D., \textbf{Macfarlane, G.S.}, Donnelly, R., \& Huntsinger, L.F. (2016). Using passive data to build an agile tour-based model: A case study in Asheville. In \textit{ Innovations in Travel Modeling Conference}. Lectern presentation. Denver, Colorado. \ifdetail Citations: 7  \fi
\item \textbf{Macfarlane, G.S.}, \& Kressner, J. D. (2016). Fusing Passive Data for Transportation Planning. In \textit{ Transportation Research Board Annual Meeting}. Poster. Washington, D.C.
\item \textbf{Macfarlane, G.S.}, \& Moreno-Cruz, J. (2015). The association between public transportation infrastructure and home price growth and stability. \textit{ In Transportation Research Board Annual Meeting}. Washington, D.C.
\item \textbf{Macfarlane, G.S.}, \& Garrow, L. A. (2012). Estimating a vehicle ownership model from targeted marketing data. In \textit{ Travel Surveys: Moving from Tradition to Practical Innovation}. Poster. Dallas, Texas.
\item Kressner, J.D., \& \textbf{Macfarlane}, G.S. (2012). Evaluating household credit reports as a replacement for episodic travel surveys. In \textit{ Transportation Research Board Annual Meeting}. Committee presentation. Washington, D.C.
\item \textbf{Macfarlane, G.S.}, Saito, M., \& Schultz, G.G. (2011). Are free right-turn channelized intersections performing as they should? In \textit{ Institute of Transportation Engineers Annual Meeting and Exhibit 2011}.
\end{myenum}



%% ==========================================
\noindent\makebox[\linewidth]{\rule{\linewidth}{0.4pt}}
\section{\secfont External \\Funding}

As Principal Investigator, totalling \$435,000:
\vspace{0.3cm}
\begin{myenum}[8]
  \item {\textbf{Macfarlane, G.S.} \& Schultz, G.G. 2024. \textit{Effectiveness of Temporary Portable Rumble Strips}. \$60,000, Utah Department of Transportation.}
  \item {\textbf{Macfarlane, G.S.} \& Brown, N.M. 2022. \textit{Activity-based Model Implementation and Analysis Considerations}. \$70,000, Utah Department of Transportation.}
  \item {\textbf{Macfarlane, G.S.} \& Schultz, G.G. 2021. \textit{Optimizing Traffic Incident Management Deployment in Utah}. \$70,000, Utah Department of Transportation.}
  \item {\textbf{Macfarlane, G.S.}, Redelfs, A.H., \& Spruance, L.A. 2021. \textit{Equitable Access to Nutrition in Utah}. \$70,000, Utah Department of Transportation.}
  \item {\textbf{Macfarlane, G.S.} 2020. \textit{ Identifying Microtransit Service Areas
through Microsimulation}. \$20,000, Utah Department of Transportation}
  \item {\textbf{Macfarlane, G.S.} 2019. \textit{ A synthesis of passive third-party datasets
used for transportation planning}. \$25,000, Utah Department of Transportation}
  \item {\textbf{Macfarlane, G.S.} 2019. \textit{ Modeling the demand and operating
characteristics for wheelchair-accessible, on-demand mobility services}.
\$60,000, Utah Department of Transportation}
  \item {\textbf{Macfarlane, G.S.} 2019. \textit{ Evaluating the Systemic Redundancy of
Critical Highway Facilities}. \$60,000, Utah Department of Transportation}
\end{myenum}

As Co-Principal Investigator, totalling \$1.44 million (\$320,000 to BYU):
\vspace{0.3cm}
\begin{myenum}[5]
  \item{Watkins, K.E. (PI), Erhardt, G.D., \& \textbf{Macfarlane, G.S.} 2024. \textit{The Potential for Behavioral Change in New High Speed Rail Lines}. \$280,000, California High Speed Rail Authority and Deutsche Bahn.}
  \item{Schultz, G.G. \& \textbf{Macfarlane, G.S.}. 2021. \textit{Analysis of
  performance measures of UDOT’s traffic incident management program: Phase
  III}. \$30,000. Utah Department of Transportation.}
  \item{Watkins, K.E. (PI), Hunter, M.S., Van Hentenryck, P., Peeta, S.,
  Brakewood, C., Cherry, C., Erhardt, G.D., \& \textbf{Macfarlane, G.S.} 2020. \textit{
  T-SCORE: Transit Serving Communities Optimally, Responsibly, and Efficiently}.
  \$1,000,000, United States Department of Transportation.}
  \item{ Schultz, G.G. (PI), \textbf{Macfarlane, G.S.} 2020. \textit{ Evaluating Signal
  Performance Measures: a Longitudinal Analysis}. \$70,000, Utah Department of
  Transportation}
  \item{ Schultz, G.G. (PI), \textbf{Macfarlane, G.S.} 2019. \textit{ Evaluating ramp meter
delay in Utah}. \$65,000, Utah Department of Transportation}
\end{myenum}

\ifdetail
Unfunded Proposals:
\vspace{0.3cm}
\begin{myenum}[3]
  \item{Dashti, S. (PI), Torres-Machi, C., Mooney, M., Misra, A., Crow, D., Mallet, S., Esteghemati, M, \textbf{Macfarlane, G.S.}, Zlatkovic, M., Kack, D. (2023). \textit{Resilient, Equitable, and Sustainable Environment through Transportation (RESET)}. \$3,000,000. United States Department of Transportation.}
  \item{Erhardt, G.D. (PI), Watkins, K.E., Brakewood, C.,  Pike, S.C., \textbf{Macfarlane, G.S.}, Chavis, C., Tal, G., Hunter, M., Van Hentenryck, P., McDonald, N. (2022). \textit{T-SCORE 2.0: Transit – Sustainable, Competitive, Responsive, and Equitable Center}. \$2,000,000. United States Department of Transportation.}
  \item{Turley Voulgaris, C.  (PI), Forsyth, A., Pandey, V., Park, H., Handy, S., \textbf{Macfarlane, G.S.}, Pande, A., Braun, L.M., Noland, R.B. (2023). \textit{Chester: Consortium for Healthy, Equitable, and Sustainable Transportation Systems for Environmental Resilience}. \$2,000,000, United States Department of Transportation}
\end{myenum}
Pending Proposals:
%\begin{myenum}[0]
%\end{myenum}

\fi
\section{\secfont{Internal Competitive Funding}}
Funded research:
\begin{itemize}
  \item {\textbf{Macfarlane, G.S.}, Guthrie, W.S., Mazzeo, B. 2021. \textit{ Measuring pavement smoothness from the perspective of e-scooters}. \$25,000, Mentored Research Grant, Brigham Young University.}
\end{itemize}

Unfunded proposals:
\begin{itemize}
  \item{\textbf{Macfarlane, G.S.}, Hooley, C., Redelfs, A., South, M. 2020 \textit{ Using Mobile Device Data to Measure Isolation and Mental Health}}.
  \$40,000, Brigham Young University Interdisciplinary Research Grant.
\end{itemize}

%% ============================================================================
\noindent\makebox[\linewidth]{\rule{\linewidth}{0.4pt}}
\section{\secfont Courses}

{\acc CCE 201: Sustainable Infrastructure}

\vspace{-.4cm}
The inter-related aspects of the different civil engineering disciplines of
environmental, geotechnical, structural, transportation, and water resources and
how they come together to develop an infrastructure system. Time value of money
and application to the infrastructure investment alternatives.


\ifdetail
\begin{tabular}{cccc}
  \toprule
  Semester & Enrolled & Student Rating (Historical) & Average GPA\\
  \midrule
  Fall 2020 & 33 & 4.1 - 4.7 (4.1)& 3.41 \\
  Fall 2021 & 64 & 4.1 - 4.5 (4.3)& 3.25 \\
  Fall 2022 & 60 & 3.6 - 4.2 (4.3)& 3.26 \\
  \bottomrule
\end{tabular}

Selected student comments:
\begin{itemize}
  \item Early in the course, Dr. Macfarlane made it clear, without really ever
    having to say so, that he cares about his students, wants us to think
    critically and ask questions, that he's really approachable and welcomes
    feedback. Overall shows high respect and high but reasonable expectations
    for students.
  \item Professor Macfarlane was always very respectful to students in the class. He
    was always willing to answer students questions or spend some time to
    further explain a concept. He was always very patient and came to class
    every day with a positive attitude and excited for class.
  \item thought he always did a good job explaining concepts in class and I thought it was very helpful to have the class notes website because it made reviewing a lot easier.
  \item I see the world a lot differently now. Different facets of sustainability are so intertwined and there's so much to consider. I feel so much more capable of making important decisions, but also more inclined to look for more perspectives.
  \item He was super open to questions, very responsive to emails, and if you asked a question and he thought of something more to add later, he'd make sure to reach out to discuss your question further. He's probably my favorite teacher this semester simply because he makes it very obvious in his actions that he wants us to succeed.
\end{itemize}


\fi

\vspace{.4cm}
{\acc CE 361: Introduction to Transportation Engineering}

\vspace{-.4cm}
Transportation systems characteristics, traffic engineering and operations,
transportation planning, geometric design, pavement design, transportation
safety, freight, public transport, sustainable transportation.

\ifdetail
\begin{tabular}{cccc}
  \toprule
  Semester & Enrolled & Student Rating (Historical) & Average GPA\\
  \midrule
  Winter 2020 & 42 & 4.4 - 4.8 (4.4) & 3.13 \\
  Winter 2021 & 38 & 4.1 - 4.7 (4.4) & 3.21 \\
  Winter 2022 & 42 & 3.9 - 4.5 (4.5) & 3.42 \\
  \bottomrule

\end{tabular}

Selected student comments:
\begin{itemize}
  \item I appreciated that he changed up the class significantly in response to midterm
  ratings. it would have been very easy to keep going with the material he was
  given, but he really made it his own and made the class so much better as a
  result.
  \item Just like the best BYU professors are. Respectful and caring about his students.
  \item He has by far been my best teacher in this whole transition to online classes situation.
  \item This class reinforced to me that I want to end up in the public sector of
transportation engineering/planning/public works. I want to work in and with
local governments to solve problems for people around me.
  \item He would take extra care to make sure that we understood the material that we were learning. Dr. Macfarlane cares a lot about his students and wants to help them in any way that he can so he takes time in lecture to answer questions.
  \item Dr. Macfarlane always seemed to have a really good understanding of the bigger picture and shared it with us, about how different facets of systems work together to make our infrastructure work great to serve people, or sometimes the opposite.
\end{itemize}


\fi

\vspace{.4cm}
{\acc CE 565: Urban Transporation Planning}

\vspace{-.4cm}
Characteristics of urban transportation planning and decision making, intermodal
transportation, land-use transportation interrelationships, transportation
demand modeling, site impact analysis, sustainable transportation, and livable
cities.

\ifdetail
\begin{tabular}{cccc}
  \toprule
  Semester & Enrolled & Student Rating (Historical) & Average GPA\\
  \midrule
  Fall 2019 & 12 & 3.9 - 4.9 (4.4) & 3.41 \\
  Fall 2020 & 19 & 4.1 - 4.7 (4.4) & 3.46 \\
  Fall 2021 & 19 & 3.9 - 4.9 (4.4) & 3.63 \\
  \bottomrule

\end{tabular}

Selected student comments:
\begin{itemize}
  \item  I'm still amazed at his quickness in responding to Slack messages, no matter
    the time of day! Dr. Macfarlane is one of the most kind professors I've had,
    he's extremely willing to work with you until you understand the concept
    that he's teaching you.
  \item The class was organized to teach us theory by allowing us to see it in action through models, which I think helped solidify those concepts.
  \item I appreciated the discussions on ethics and applying them to our work - few of my classes have mentioned them more than once in a semester, and rarely with applicable concerns.
  \item I have more of a desire to be involved in the planning process in the future communities that I'll live in.
  \item Dr. Macfarlane is the reason that I didn't change majors. CE 361 was my favorite fundamentals of engineering course and made me want to become an engineer. CE 565 (this course) has been my favorite course at BYU. Dr. Macfarlane knows his stuff and does a good job of teaching in an interesting and engaging way.
\end{itemize}

\fi

\vspace{.4cm}
{\acc CE 594R: Data Science for Engineers}

\vspace{-.4cm}
A first-semester graduate course in programming and data science techniques:
literate programming in Markdown and LaTeX, version control with git, data
manipulation and visualization with R, object-oriented programming with Java.

\ifdetail
\begin{tabular}{cccc}
  \toprule
  Semester & Enrolled & Student Rating (Historical) & Average GPA\\
  \midrule
  Fall 2019 & 4 & 4.8 - 5.0 () & 3.85 \\
  Fall 2020 & 9 & 3.5 - 4.7 () & 3.81 \\
  Fall 2021 & 6 & 4.5 - 5.0 () & 3.68\\
  \bottomrule

\end{tabular}

Selected student comments:
\begin{itemize}
  \item I know you didn't have to teach this course, but I appreciate you doing it.
    When I signed up for the course, I thought I would learn about data science
    techniques (as what I should look for in analyzing data--types of tests to
    perform, processes, etc). I think that could be a good edition that could
    come from Dr. Macfarlane, is teaching more about data science and what that
    entails, rather than just use R to solve problems.
    \item I felt like he was pretty good at responding to emails and being merciful and helpful and understanding. He definitely recognized that this semester was different and was flexible.
    \item I really appreciated Dr. Macfarlane and he did curate some great resources for us to use. He was willing and available to help. And he was flexible and merciful in the midst of a challenging and different semester.
    \item Almost every class period something that Dr. Macfarlane taught blew my mind! This was an excellent course and much needed to round out my engineering education.
\end{itemize}

\fi



\vspace{.4cm}
{\acc CE 662: Transport Simulation and Analysis}

\vspace{-.4cm}
An advanced graduate course in traffic flow theory and simulation. Topics
include shock wave analysis, discrete event simulation of queues and daily
activity pattern choices, car following models, and traffic simulation.
Laboratory assignments use MATSim and PTV Vissim simulation softwares.


\ifdetail
\begin{tabular}{cccc}
  \toprule
  Semester & Enrolled & Student Rating (Historical) & Average GPA\\
  \midrule
  Winter 2019 & 2 & 4.6 (4.3) & 3.70 \\
  Winter 2020 & 3 & 5.0 - 5.0 (4.4) & 3.00 \\
  Winter 2022 & 7 & 3.3 - 5.0 (4.3) & 3.44 \\
  \bottomrule
\end{tabular}

Selected student comments:
\begin{itemize}
\item  Dr. Macfarlane is really patient in explaining the challenging concepts, and he will find the simple way to describe or teach twice to make sure we totally understand.
\item Always came well prepared to lectures. It was obvious that he had put time,
  thought, and effort into preparing an engaging and effective lecture, every
  time.
\item The instructor expected us to use critical thinking and to not just go for the
  "right" answer without thought about it. I think this is a valuable way of
  building students' character.
\item The course was well organized because it was clearly explained how new
  material builds from the previous subject. It was easy to see the progression
  of how each lecture builds from the last and prepares you for the next.
\end{itemize}
\fi


\vspace{.4cm}
{\acc CE 694R: Advanced Choice Modeling}

\vspace{-.4cm}
An advanced graduate course in discrete choice modeling. Theory of choice
models, including estimation and validation techniques. Mode choice models for
work and non-work trip purposes using multinomial and nested logit models.

\ifdetail
\begin{tabular}{cccc}
  \toprule
  Semester & Enrolled & Student Rating (Historical) & Average GPA\\
  \midrule
  Winter 2021 & 5 & 4.0 - 5.0 () & 3.48 \\
  \bottomrule
\end{tabular}


\fi

% {\acc CEE 6622: Travel Demand Analysis}\hfill{Ga. Tech: Spring 2014}
%
% \vspace{-.4cm}
% This course teaches graduate students to develop and use urban travel
% demand models, including trip-based and activity-based modeling methods and
% experience with a modern practicing regional model.

%% ============================================================================
\noindent\makebox[\linewidth]{\rule{\linewidth}{0.4pt}}
\section{\secfont Graduate Mentoring}
Students mentored as graduate committee chair (9 total):
\vspace{0.3cm}
\begin{myenum}[9]
  \item Emily Youngs. \textit{Exploring the Link between Travel Behavior and Mental Health}. MS granted August 2024.
  \item Hayden Atchley. \textit{A Comparative Illustration of Trip- and Activity-Based Modeling Techniques}. MS granted August 2024.
  \item Daniel Jarvis. \textit{Simulating Incident Management Team Response and Performance}. MS granted December 2023.
  \item Natalie Gray, \textit{Evaluating parameter uncertainty in transportation demand models}. MS granted June 2023.
  \item Emma Stucki, \textit{Evaluating equitable access to nutrition in Utah}. MS granted December 2022.
  \item Gillian Riches, \textit{Transforming GPS points to daily activities using simultaneously optimized DBSCAN-TE parameters}. MS granted December 2022.
  \item Christopher Day, \textit{Forecasting ride-hailing across multiple model frameworks.} MS granted December 2022.
  \item Max Barnes, \textit{Resiliency of Utah's road network: a logit-based approach.} MS granted December 2021.
  \item Nate Lant, \textit{Estimation and simulation of daily activity patterns for individuals using wheelchairs.} MS granted June 2021.
\end{myenum}

Students mentored as graduate committee member (22 total, 5 current, 3 non-BYU):
\vspace{0.3cm}
\begin{myenum}[22]
  \item Sara Jaen Hosey, Ph.D.\ in Civil Engineering pre-proposal.
  \item Samuel McKinnon, Ph.D.\ in Mechanical Engineering pre-proposal.
  \item Ian MacGregor. MS scheduled December 2024.
  \item Adam Hill. MS scheduled December 2024.
  \item Sam Runyan. MS scheduled December 2024.
  \item Joel Hyer, \textit{Analysis of benefits of UDOT's exapanded incident management team program.} MS granted April 2024.
  \item Matthew Davis, \textit{Effectiveness of Intelligent Transportation Systems on Utah Roadways}. MS granted December 2023.
  \item William Charlton, \textit{Web-based data visualization in support of agent-based microsimulation models}. Ph.D. granted October 2023 (at TU Berlin).
  \item Wang Bangyu (Bruce), \textit{Evaluating and advancing automated traffic signal performance measures: Statistical and machine learning approaches }. Ph.D.\ granted August 2023.
  \item Tomas Barriga, \textit{Using severity weighted risk scores to prioritize safety funding in Utah}. MS granted August 2023.
  \item Benjamin Meek, \textit{Load-deformation behavior of tension-only X-brace roof truss diaphragms.} MS granted April 2023.
  \item Mylan Cook. \textit{Physics-guided modeling of acoustic environments using limited spatio-spectro-temporal data}. Ph.D.\ in Physics granted June 2023.
  \item Cory Ward, \textit{An evaluation of the safe speed limit setting procedure and tool for Utah} (Project). MS granted December 2022.
  \item Samantha Lau, \textit{Analysis of using V2X DSRC equipped snowplows to request signal preemption.} MS granted August 2022.
  \item Tanner Daines, \textit{Evaluating ramp meter delay in Utah}. MS granted April 2022.
  \item Logan Bennett, \textit{Analysis of benefits of an expansion to UDOT’s incident management program}. MS granted August 2021.
  \item Camille Lunt, \textit{Crash analysis methodology for segments of Utah highway}. MS granted April 2021.
  \item Chad Vickery, \textit{Quantifying the conditioning period for geogrid-reinforced aggregate base materials through cyclic loading}. MS granted August 2020.
  \item Michael Sheffield, \textit{Impacts of changing the transit signal priority requesting threshold on bus performance and general traffic: a sensitivity analysis}. MS granted June 2020.
  \item Michael Adamson, \textit{An analysis of decision boundaries for left-turn treatments}. MS granted April 2019.
  \item Nico Boyd, \textit{Accessibility to urban parks and health outcomes on the neighborhood level}. MS granted August 2018 (at Georgia Tech).
  \item Zhang Bingling, \textit{Friendship influences on air travel: a social autoregressive analysis}. MS granted August 2014 (at Georgia Tech).
\end{myenum}


\noindent\makebox[\linewidth]{\rule{\linewidth}{0.4pt}}
\section{\secfont Undergraduate Mentoring}

Students mentored on funded research projects (25 total):
\vspace{0.2cm}
\begin{myenum}[25]
  \item Kaleigh Squires, undergraduate research assistant in demand microsimulation (2024 - )
  \item Kamryn Mansfield, undergraduate research assistant in demand microsimulation (2023 - 2024).
  \item Harrison Holdsworth, undergraduate research assistant in demand microsimulation (2021 - 2022).
  \item Brynn Woolley, undergraduate research assistant in demand microsimulation (2022 - 2024). Ph.D.\ Student at University of Michigan.
  \item Jeremy Raine, undergraduate research assistant in community resources (2022). Majoring in Psychology.
  \item Jonathan Orton, undergraduate research assistant in e-scooters and pavements (2022).
  \item Dylan Apelu, undergraduate research assistant in e-scooters and pavements (2021 - 2023). Now MS student at Georgia Institute of Technology.
  \item Hayden Atchley, undergraduate research assistant in demand microsimulation (2020 - 2022). Now MS student at BYU.
  \item Nicole Adams, undergraduate research assistant in e-scooters and pavements (2021 - 2022). 
  \item Liv Neeley, undergraduate research assistant in e-scooters and pavements (2021 - 2022). 
  \item Kaeli Monahan, undergraduate research assistant in community resources and passive data (2020 - 2022). Now BS student in Mechanical Engineering at BYU.
  \item Shannon Anderson, undergraduate research assistant in V2X data, jointly mentored with Grant Schultz (2020 - 2022). Now working in industry.
  \item Corey Ward, undergraduate research assistant in ramp meter evaluation, jointly mentored with Grant Schultz (2020 - 2021). Completed MS at BYU.
  \item Michael Copley, undergraduate research assistant in third-party passive data (2018 - 2021). Completed MS at University of Illinois.
  \item James Umphress, undergraduate research assistant in ramp meters (2020-2021), jointly mentored with Grant Schultz. Completed MS at Oregon State University.
  \item Christopher Day, undergraduate research assistant in demand microsimulation (2020 - 2021). Completed MS at BYU.
  \item Emma Stucki, undergraduate research assistant in community resources (2020 - 2021). Completed MS at BYU.
  \item Gillian Martin Riches, undergraduate research assistant in community resources (2020 - 2021). Completed MS at BYU.
  \item Natalie Gray, undergraduate research assistant in network resiliency (2019 - 2021). Completed MS at BYU.
  \item Max Barnes, undergraduate research assistant in network resiliency (2019 - 2020). Completed MS at BYU.
  \item Kim Munseok, undergraduate research assistant in demand microsimulation (2020 - 2021). Completed BS in Computer Science at BYU.
  \item Christian Hunter, undergraduate research assistant in demand microsimulation (2018 - 2019). Completed MS at University of Texas at Austin.
  \item Christian Vanderhoeven, undergraduate research assistant in demand microsimulation (2019). Completed MS at University of Washington.
  \item Hayden Anderson, undergraduate research assistant in e-scooters (2019 - 2020). Completed MS at University of California, Davis.
  \item Emily Andrus, undergraduate research assistant in signal peformance data, jointly mentored with Grant Schultz (2019).
  \item Sabrina McCuen, undergraduate research assistant in signal peformance data, jointly mentored with Grant Schultz (2019 - 2020).
\end{myenum}

Honors students mentored as department honors coordinator (3 total).
\vspace{0.2cm}
\begin{myenum}[3]
  \item Becca Apgar, \textit{Development and demonstration of an apparatus for assessing frost-heave susceptibility of soil} (2024).
  \item Daria Sofia Velasco-Vega, \textit{Thermal performance of thin-shell concrete dome structures} (2023).
  \item Emma Kratz-Bailey, \textit{Accessible methods, novel arrangement: Developing self-centering composite structural frames for highly resilient buildings} (2022).
\end{myenum}



Students mentored as civil engineering capstone team advisor (23 total):
\vspace{0.2cm}
\begin{description}
  \item[2023-2024] BYU household travel survey. Sponsored by BYU Sustainability Office. Students: 
    Megan Hungerford, Ellie Johns, Kamryn Mansfield, and Myrranda Salmon.
  \item[2022-2023] SR-140 Corridor alternatives. Sponsored by Bluffdale City. Students:
    Clinton Childers, Robert Mickelson, Trevor Mickelson, and Joseph Wells.
  \item[2021-2022] BYU household travel survey. Sponsored by BYU Sustainability Office. Students: Nicole Adams, Hayden Atchley, Kyle Leatham, and Daniel Jarvis.
  \item[2020-2021] Forecasting demand for future FrontRunner scenarios. Sponsored by Utah Transit Authority. Students: Gillian Martin Riches, Tomas Barriga, Landon Pratt, and Cole Larsen.
  \item[2019-2020] UTA microtransit pilot evaluation. Sponsored by Utah Transit Authority. Students: Christian Hunter, Austin Martinez, and Elizabeth Smith.
  \item[2018-2019] Demand for wheelchair-accessible vehicles. Sponsored by Utah Transit Authority. Students: Nate Lant, Byron Yates, Cody Irons, and Matthew Strong.
\end{description}


% \section{\secfont Professional \\Experience}
% {\acc Singapore Mission of the Church of Jesus Christ of Latter-day Saints}
%
% \vspace{-.3cm}
% \textit{ Missionary} \hfill {June 2004 - June 2006}\\
% Served as volunteer cleric in Singapore, Malaysia, and Sri Lanka.



%% ============================================================================
\noindent\makebox[\linewidth]{\rule{\linewidth}{0.4pt}}
\section{\secfont Awards and\\ Honors}
\begin{description}
\item[\acc Most Influential Faculty] Given to the faculty member in the Civil Engineering program
whom graduating seniors name as the most influential on their undergraduate
education. Awarded by 2022 graduating class.
\item{\acc TUM Global Visiting Professor} Selected to enrich the vibrant research culture at 
the Technische Universit\"at M\"unchen by virtue of innovative approaches and to explore new, cutting-edge research fields. Awarded in 2023.
\item[\acc ASCE ExCEEd Teaching Fellow] Participated in week-long intensive teacher
development program. Class of 2022.
\item[\acc Dwight David Eisenhower Graduate Fellowship] Full doctoral funding from
the Federal Highways Administration 2011-2013, one of five awards nationally.
Awarded supplemental grant in 2013.
\item[\acc Eno Center for Transportation Leadership Development Conference]
Participated in the 2012 program; nominated by the Ivan Allen, Jr.\ College of
Liberal Arts at Georgia Tech.
\item[\acc Parsons Brinckerhoff - Jim Lammie Engineering Scholarship] Awarded
 to the top engineer in the 2011 American Public Transportation Foundation (APTF)
competition. Sponsored by Mike Allegra, general manager of the Utah Transit
Authority. Renewed in 2012.
% Lesser awards (comment out for space saving)
\item[\acc Gordon W. Schultz Graduate Fellowship] Given to the Georgia Tech
student in travel demand modeling who exhibits innovation, problem-solving, and
practical application.
\item[\acc National Science Foundation Graduate Fellowship Program] Honorable
Mention in 2011 and 2012, as a first- and second-year graduate student.
\item[\acc Jim McGee Memorial Scholarship] Cash award from the Georgia chapter of the
American Society of Highway Engineers, one of two awards in 2011.
\item[\acc Georgia Department of Transportation Scholarship] One of ten cash awards in
2010 to students from the southeastern United States.
\item[\acc Office of Research and Creative Activities (ORCA) Grant] Competitive
research grant to survey Chinese transportation planning practices, one of
several undergraduate research awards from Brigham Young University.
\item[\acc Freeman-Asia Award] Grant to study Chinese finance and globalized
engineering at Nanjing University in the People's Republic of China from the
Institute for International Education.
\end{description}

%% ============================================================================
\noindent\makebox[\linewidth]{\rule{\linewidth}{0.4pt}}
\section{\secfont External \\ Citizenship}



Panel Member, NCHRP 08-184, \textit{Framework for Assessing Induced Demand Effects of Various Roadway Investments}. (2023 - )

Member, Provo City Transportation and Mobility Advisory Commission

Transportation Research Board of the National Academies of Science:

\begin{itemize}
  \item AEP50: Travel Demand Forecasting Member of the committee (2019 --- ) on
  travel demand forecasting. Chair of the travel forecasting resources
  subcommittee and editor of \url{tfresource.org}.
  \item AMS20: Economics and Land Development Member of the committee (2014
  --- 2022). formerly standing committee on transportation and land use.
  \item Young Members Council (2019 --- 2021). Planning and Environment subcommittee
  chair.
\end{itemize}


Reviewer for the following journals:

\begin{itemize}
  \item Transportation Research Part A: Policy and Practice
  \item Transportation Research Record
  \item Environment and Planning B: Urban Analytics and City Science
  \item International Journal of Sustainable Transport
  \item Journal of Public Transportation
\end{itemize}

Member of the following professional organizations:

\begin{itemize}
  \item American Society of Civil Engineers (2022 - )
  \item Zephyr Foundation (2020 - 2022).
  \item Institute of Transportation Engineers (2009-2013, 2018-2020)
  \item Tau Beta Pi (Utah $\beta$ '09).
  \item Young Professionals in Transportation (2013-2018); organizing co-chair of
Triangle NC chapter.
  \item American Public Transportation Association scholar task force (2011 - 2013).
\end{itemize}

\noindent\makebox[\linewidth]{\rule{\linewidth}{0.4pt}}

\section{\secfont Media}
DeBrule, D. (2024). E-scooter safety top of mind after mother's death. \textit{Fox 13 Salt Lake City}. Quoted expert opinion. March 27th, 2024. \url{https://www.fox13now.com/news/local-news/e-scooter-safety-top-of-mind-after-mothers-death}

Carlisle, N. (2022). Here’s what it might cost to ride the Little Cottonwood Canyon gondola. \textit{Fox 13 Salt Lake City}. Quoted expert opinion. November 16th, 2022. \url{https://www.fox13now.com/news/fox-13-investigates/heres-what-it-might-cost-to-ride-the-little-cottonwood-canyon-gondola}.

McCann, A. (2021). Best and worst cities to drive in. \textit{WalletHub}. Quoted expert opinion. August 31, 2021. \url{https://wallethub.com/edu/best-worst-cities-to-drive-in/13964#expert=Gregory_Macfarlane}

\textbf{Macfarlane, G.S.}. (2020). No, Utah County does not have to choose between preservation and growth. \textit{Deseret News}. Guest Opinion, August 21, 2020. \url{https://www.deseret.com/opinion/2020/8/21/21376479/}

\noindent\makebox[\linewidth]{\rule{\linewidth}{0.4pt}}
\section{\secfont Internal \\ Citizenship}
\begin{description}
  \item[Department honors coordinator] (2019 --- ). Encourage students to participate
  in the honors program, and participate on honors thesis committees in the
  department.
  \item[Department undergraduate committee] Chair (2023 --- ), Member (2021 --- ). Leading curriculum revisions for civil engineering program.
  \item[Department faculty development and capital improvement committee] (2018 --- 2021).
\end{description}


%% ==========================================

% \noindent\makebox[\linewidth]{\rule{\linewidth}{0.4pt}}
% \section{\secfont Skills}
% Significant computer software and programming ability, including expert skills
% in R, \LaTeX, and git. Also experienced with Java, C, Python, QGIS,
% Cube/Voyager,  Matlab, PTV Vissim / Visum, TransCAD (with GISDK).


%\clearpage
%\thispagestyle{empty}
%\section{\secfont References}
%\textit{ Rick Donnelly, Ph.D., AICP} \\
%6100 Uptown Boulevard NE, Suite 700, Albuquerque, NM 87110 \\
%DonnellyR@pbworld.com 1.505.878.6524
%
%\textit{ Leta F. Huntsinger, Ph.D., PE} \\
%434 Fayetteville Drive, Suite 1500, Raleigh NC 27601 \\
%Huntsinger@pbworld.com 1.919.836.4086
%
%%\textit{ Laurie A. Garrow, Ph.D.} \\
%%790 Atlantic Drive, Atlanta GA 30332-0355\\
%%laurie.garrow@ce.gatech.edu 1.404.385.6634
%
%\textit{ Juan Moreno-Cruz, Ph.D.} \\
%221 Bobby Dodd Way, Atlanta GA 30332\\
%juan.moreno-cruz@econ.gatech.edu 1.404.385.1100
%
%\textit{ Kari E. Watkins, Ph.D., P.E.} \\
%790 Atlantic Drive, Atlanta GA 30332-0355\\
%kari.watkins@ce.gatech.edu 1.206.250.4415
%
%%\textit{ Patricia L. Mokhtarian, Ph.D.} \\
%%790 Atlantic Drive, Atlanta GA 30332-0355\\
%%patmokh@ce.gatech.edu 1.404.385.1443
%
%%\textit{ Patrick S. McCarthy, Ph.D.} \\
%%221 Bobby Dodd Way, Atlanta GA 30332\\
%%mccarthy@gatech.edu 1.404.894.4914
%
%%\textit{ Mitsuru Saito, Ph.D., P.E.} \\
%368J Clyde Building, Provo UT 84602\\
%msaito@byu.edu 1.801.422.6326
\end{resume}
%\vfill\moveleft.25\hoffset\centerline{References are available upon request.}



\end{document}
