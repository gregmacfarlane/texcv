\documentclass[margin,line]{res}

%This lets me make hyper references to BYU and Georgia Tech
\usepackage{booktabs}
\usepackage{enumitem}
\usepackage[colorlinks=false]{hyperref}
\oddsidemargin -.5in
\evensidemargin -.5in
\textwidth=6.0in
\itemsep=0in
\parsep=0in

%% ==========================================
%This allows me to use Bibtex to create my publication list
%\usepackage{natbib}
%\makeatletter
%\renewenvironment{thebibliography}[1]
%     {%\section*{\bibname}% <-- remove this to make section naming correct
%      \@mkboth{\MakeUppercase\bibname}{\MakeUppercase\bibname}%
%      \list{\@biblabel{\@arabic\c@enumiv}}%
%           {\settowidth\labelwidth{\@biblabel{#1}}%
%            \leftmargin\labelwidth
%            \advance\leftmargin\labelsep
%            \@openbib@code
%            \usecounter{enumiv}%
%            \let\p@enumiv\@empty
%            \renewcommand\theenumiv{\@arabic\c@enumiv}}%
%      \sloppy
%      \clubpenalty4000
%      \@clubpenalty \clubpenalty
%      \widowpenalty4000%
%      \sfcode`\.\@m}
%     {\def\@noitemerr
%       {\@latex@warning{Empty `thebibliography' environment}}%
%      \endlist}
%\makeatother

\usepackage{bibentry}
\nobibliography*
\makeatletter
\def\bibliography#1{%
   \if@filesw
   \immediate\write\@auxout{\string\bibdata{#1}}%
\fi}
\makeatother
%\usepackage[resetlabels]{multibib}
%  \newcites{j,c,w}{{Test},{Test1},{Test2}}
% j- journals c-conferences p-presentations
\setdescription{font=\normalfont}

%% ==========================================
% This defines the list types and formatting
\newenvironment{list1}{
  \begin{list}{\ding{113}}{%
      \setlength{\itemsep}{0in}
      \setlength{\parsep}{0in} \setlength{\parskip}{0in}
      \setlength{\topsep}{0in} \setlength{\partopsep}{0in}
      \setlength{\leftmargin}{0.17in}}}{\end{list}}
\newenvironment{list2}{
  \begin{list}{$\bullet$}{%
      \setlength{\itemsep}{0in}
      \setlength{\parsep}{0in} \setlength{\parskip}{0in}
      \setlength{\topsep}{0in} \setlength{\partopsep}{0in}
      \setlength{\leftmargin}{0.2in}}}{\end{list}}
% define fonts
\usepackage{fontspec}
%\setmainfont[Mapping=tex-text]{Palatino}
%\newfontface\acc{Marcellus SC}
\newcommand{\secfont}{\scshape }
\newcommand{\acc}{\scshape }

\begin{document}
% Center the name over the entire width of resume:
 \moveleft.5\hoffset\centerline{\LARGE\scshape Gregory S.  Macfarlane}
\vspace{.05in}
 \moveleft.5\hoffset\centerline{Brigham Young University}
 \moveleft.5\hoffset\centerline{
	 \href{mailto:gregmacfarlane@byu.edu}{gregmacfarlane@byu.edu}
   801.422.8505}
\vspace{.05in}
% address begins here
% Again, the address lines must be centered over entire width of resume:
 \moveleft.5\hoffset\centerline{430 Engineering Building}
 \moveleft.5\hoffset\centerline{Provo, UT 84602}

\begin{resume}
%\section{\sc Contact\\ Information}
\vspace{.05in}

%% ============================================================================

%% ============================================================================
\section{\secfont Education}
\href{http://www.gatech.edu}{\acc Georgia Institute of Technology}
\\
	\vspace*{-.1in}
	\begin{list1}
	\item[] Ph.D., Transportation Systems Engineering \hfill May 2014
	\begin{list2}
  	\item[] Advisor: Laurie A. Garrow
		\item[] Dissertation: ``Using Big Data to Model Travel Behavior: Applications to Vehicle Ownership and Willingness-to-Pay for Transit Accessibility''
%		\item[] GPA: 3.86/4.0
%		\item[] Committee: Laurie A. Garrow (Chair - CEE), Juan Moreno-Cruz
%(Economics), \\ Patricia L. Mokhtarian (CEE), Kari E. Watkins (CEE), Patrick S.
%McCarthy (Economics), Jeffrey P. Newman (CEE)
  \end{list2}
	\vspace*{.05in}
	\item[]M.S., Economics
	\end{list1}

\href{http://www.byu.edu}{\acc Brigham Young University}
\\
\vspace*{-.1in}
\begin{list1}
  \item[] B.S. with University Honors, Civil Engineering\hfill December 2009
	\item[] Minor	degrees in Mathematics and Asian Studies
%  \begin{list2}
%		\item[] Honors Thesis: ``Delay Patterns
%			and Perceptions at Free Right-turn Channelized Intersections''
%		\item[] Advisor: Mitsuru Saito
%%		\item[] GPA: 3.7/4.0
%    \item[] Studies Abroad: Nanjing, China (Engineering);
%			Naples, Italy (Arch{\ae}ology)
%	\end{list2}
\end{list1}

%\textit{ Research Interests ---} Passive data and its applications in transport and
%land use modeling and forecasting.

%% ============================================================================
\noindent\makebox[\linewidth]{\rule{\linewidth}{0.4pt}}
\section{\secfont Professional \\ Experience}
{\acc Brigham Young University}

\vspace{-.4cm}
\textit{Assistant Professor} \hfill {November 2018 - }\\
Researching the application of passive data sets in transport and land use
modeling, including spatial and social effects on travel behavior.

{\acc Transport Foundry} Atlanta, Georgia

\vspace{-.3cm}
\textit{Transportation Engineer} \hfill {April 2017 --- October 2018}\\
Developed a data-driven travel demand model from passive data sources.

{\acc WSP | Parsons Brinckerhoff} Raleigh, North Carolina

\vspace{-.3cm}
\textit{Technical Principal, Systems Analysis Group} \hfill {June 2014 --- April 2017}\\
Developed advanced travel demand models for public sector clients.

{\acc University of North Carolina, Chapel Hill}

\vspace{-.4cm}
\textit{Adjunct Lecturer/Teaching Assistant} \hfill {January 2017 - May 2017}\\
Lectured on transportation data, discrete choice econometrics, and mode choice
models in the graduate travel demand analysis course.

{\acc Georgia Institute of Technology}

\vspace{-.4cm}
\textit{Post-doctoral Researcher} \hfill {January 2014 - May 2014}\\
Developed a curriculum to teach sustainable transportation engineering and
analysis, in partnership with the National Center for Sustainable
Transportation.


{\acc Utah Transit Authority} Salt Lake City, Utah

\vspace{-.3cm}
\textit{Strategic Planning Intern} \hfill {May 2009 - June 2010}\\
Developed transit operating scenarios for the Wasatch Front long-range
transportation plan and for UTA's internal scenario planning and programming purposes.

{\acc Hales Engineering} Lehi, Utah

\vspace{-.3cm}
\textit{Engineering Intern} \hfill {July 2008 - May 2009}\\
Prepared traffic impact analyses for commercial and residential developments.


\section{\secfont Research\\ Interests}
Transportation planning and engineering, travel demand modeling,
application of passive data products in transportation planning and forecasting.


%% ==========================================
% BibTex reference management

% Journal Articles
%\section{\secfont Under \\ Review}
%\bibentry{epb_parkaccess}

\noindent\makebox[\linewidth]{\rule{\linewidth}{0.4pt}}

\section{\secfont Refereed \\ Journal\\ Articles}
First author or first faculty author on 5 of 10 total journal articles.
$^\dagger$indicates BYU graduate student authors, $^*$indicates BYU undergraduate authors.
\vspace{.3cm}
\begin{enumerate}[left= 0pt, itemsep=12pt, label=A\arabic*]
  \item \textbf{Macfarlane, G.S.}, Sheffield, M.H.$^\dagger$, Bennet, L.S.$^\dagger$, \& Schultz, G.G. (2021).
  The Effect of Transit Signal Priority on Bus Rapid Transit Headway Adherence. \textit{Findings}, June. \url{https://doi.org/10.32866/001c.24499}.

  \item\textbf{Macfarlane, G.S.}, Hunter, C.$^*$, Martinez, A.$^*$, \& Smith, E.$^*$  (2021). Rider
Perceptions of an On-Demand Microtransit Service in Salt Lake County, Utah
\textit{ Smart Cities} 4(2): 717-727. \url{https://doi.org/10.3390/smartcities4020036}

  \item\textbf{Macfarlane, G.S.}, Boyd, N., Taylor, J.E., \& Watkins, K. (2021) Modeling the impacts of park access
on health outcomes: A utility-based accessibility approach. \textit{ Environment and
Planning B: Urban Analytics and City Science}, 48(8), 2289–2306. \url{https://doi.org/10.1177/2399808320974027}

  \item Glenn, J., Bluth, M.$^*$, Christianson, M.$^*$, Pressley, J.$^*$, Taylor, A.,
\textbf{Macfarlane, G.S.}, \& Chaney, R. A. (2020). Considering the Potential Health
Impacts of Electric Scooters: An Analysis of User Reported Behaviors in Provo,
Utah. \textit{ International Journal of Environmental Research and Public Health},
17(17), 6344. \url{https://doi.org/10.3390/ijerph17176344}

  \item\textbf{Macfarlane, G.S.}, Garrow, L.A., \& Moreno-Cruz, J. (2015). Do Atlanta
residents value MARTA? Selecting an autoregressive model to recover willingness
to pay. \textit{ Transportation Research Part A: Policy and Practice}, 78, 214–230.
\url{https://doi.org/10.1016/j.tra.2015.05.010}

  \item\textbf{Macfarlane, G.S.}, Garrow, L.A., \& Mokhtarian, P. L. (2015). The influences of
past and present residential locations on vehicle ownership decisions.
\textit{ Transportation Research Part A: Policy and Practice}, 74, 186–200.
\url{https://doi.org/10.1016/j.tra.2015.01.005}

  \item Brakewood, C., \textbf{Macfarlane, G.S.}, \& Watkins, K.E. (2015). The impact of
real-time information on bus ridership in New York City. \textit{ Transportation Research
Part C: Emerging Technologies}, 53, 59–75. \url{https://doi.org/10.1016/j.trc.2015.01.021}

  \item Binder, S., \textbf{Macfarlane, G.S.}, Garrow, L.A., \& Bierlaire, M. (2014).
Associations among household characteristics, vehicle characteristics and
emissions failures: An application of targeted marketing data. \textit{ Transportation
Research Part A: Policy and Practice}, 59, 122–133.
\url{https://doi.org/10.1016/j.tra.2013.11.005}

  \item Wall, T.A., \textbf{Macfarlane, G.S.}, \& Watkins, K.E. (2014). Exploring the use of
egocentric online social network data to characterize individual air travel
behavior. \textit{ Transportation Research Record}, 2400, 78–86.
\url{https://doi.org/10.3141/2400-09}

  \item McBride, J.H., Keach, R. W., Macfarlane, R.T., De Simone, G.F., Scarpati, C.,
Johnson, D.J., \textbf{Macfarlane, G.S.}, \& Weight, R.W.R. (2009). Subsurface visualization using
ground-penetrating radar for archaeological site preparation on the northern
slope of Somma-Vesuvius: a Roman site, Pollena-Trocchia, Italy. \textit{ Il Quaternario,
Italian Journal of Quaternary Sciences}, 22(1), 39–52. \url{https://portal.issn.org/resource/ISSN/0394-3356}
\end{enumerate}

\section{\secfont{Venue Notes}}
\begin{description}
  \item[\textit{Transportation Research Part A: Policy and Practice}] is
   a leading international journal with robust peer review focusing on
   transportation policy  analysis and the planning of transportation systems.
   CiteScore: 8.5; 17/318 in civil engineering. Impact factor: $5.594$. Publisher: Elsevier.
  \item[\textit{Environment and Planning B: Urban Analytics and City Science}]
  is a leading international journal with robust peer review publishing
  cutting-edge research in analytical methods for urban planning and
  design. CiteScore 4.6; 6/138 in architecture. 5-year impact factor: $3.889$. Publisher: Sage.
  \item[\textit{Transportation Research Part C: Emerging Technologies}] is a
  leading international journal with robust peer review focusing on applications
  and implications of technology in transportation systems.
  CiteScore: 14.0; 3/318 in civil engineering. 5-year impact factor: $8.089$. Publisher: Elsevier.
  \item[\textit{Findings}] is an interdisciplinary, independent, community-led,
  peer-reviewed, open access journal focused on short, clear, and pointed
  research results.  The journal was established in 2019. Publisher: University of
  Sydney and McGill University.
  \item[\textit{International Journal of Environmental Research and Public Health}]
  is an interdisciplinary, peer-reviewed, open access journal. CiteScore: 3.4;
  179/526 in public health.  5-year impact factor: $2.789$. Publisher: MDPI.
  \item[\textit{Smart Cities}] is an international, scientific, peer-reviewed,
  open access journal on the science and technology of smart cities. This
  journal was established in 2019. Publisher: MDPI.
\end{description}


\section{\secfont{Under Review}}

\begin{enumerate}[left= 0pt, itemsep=12pt, label=U\arabic*]
\item Wang, B.$^\dagger$, Schultz, G.G., \textbf{Macfarlane, G.S.}, \& McCuen, S.$^*$ (2021). Evaluating signal systems using automated traffic signal performance measures. Under initial review at \textit{Journal of Traffic and Transportation Engineering}.

\item \textbf{Macfarlane, G.S.}, Turley Voulgaris, C., \& Tapia, T. (2021). If you build it who will come? Equity analysis of park system changes during COVID-19 using passive origin-destination data. Under initial review at \textit{Journal of Transportation and Land Use}.

\item Turley Voulgaris, C., \textbf{Macfarlane, G.S.}, Kaylor, J., Su, T., Bauranov, A. (2021). Whose emissions are these anyway? Estimating vehicle miles traveled to account for site-level climate impacts. Under initial review at \textit{Journal of Planning Education and Research}.
\end{enumerate}
% \textbf{Macfarlane, G.S.}, \& Lant, N.$^\dagger$ (2021). Identifying Segmentation Strategies in a Daily Activity Pattern Model for Wheelchair Users.

\noindent\makebox[\linewidth]{\rule{\linewidth}{0.4pt}}
\section{\secfont Peer- Reviewed Conference \\ Papers}
All listed conference publications are full papers and include at least a
single-blind review process with multiple expert reviewers for consideration.
\vspace{.3cm}
\begin{enumerate}[left=0pt, itemsep=12pt, label=C\arabic*]
\item Turley Voulgaris, C., \textbf{Macfarlane, G.S.}, Kaylor, J., Su, T., Bauranov, A. (2022). Whose emissions are these anyway? Estimating vehicle miles traveled to account for site-level climate impacts. In \textit{Transportation Research Board Annual Meeting}. Washington, D.C.

\item \textbf{Macfarlane, G.S.}, Copley, M.$^*$, \& Stucki, E.$^\dagger$. (2021). Utility-Based Accessibility to Community Resources: An Application of Location-Based Services Data. In \textit{North American Regional Science Conference}. Denver, Colorado.

\item \textbf{Macfarlane, G.S.}, \& Tapia, T. (2020). Developing a Park Activity Location Choice Model from Passive Origin-Destination Data Tables. In \textit{ Transportation Research Board Annual Meeting}. Washington, D.C.

\item Zhang, B., \textbf{Macfarlane, G.S.}, Wall, T.A., \& Watkins, K.E. (2014). Friendship Influences on Air Travel: A Social Autoregressive Analysis. In \textit{ North American Regional Science Conference}. Washington, D.C.: Regional Science Association International.

\item \textbf{Macfarlane, G.S.}, Moreno-Cruz, J., \& Garrow, L. A. (2013). Does Atlanta value MARTA? Selecting an autoregressive model to recover willingness-to-pay. In \textit{ North American Regional Science Conference}. Atlanta, Georgia.

\item \textbf{Macfarlane, G.S.}, Saito, M., \& Schultz, G.G. (2011). Delay underestimation at free right-turn channelized intersections. In \textit{ 6th International Symposium on Highway Capacity and Quality of Service} (Vol. 16, pp. 560–567). https://doi.org/10.1016/j.sbspro.2011.04.476

\item \textbf{Macfarlane, G.S.}, Saito, M., \& Schultz, G.G. (2011). Driver perceptions at free right-turn channelized intersections. In \textit{ T\&DI Congress 2011: Integrated Transportation and Development for a Better Tomorrow} (Vol. 398, pp. 108–108). ASCE. https://doi.org/10.1061/41167(398)108
\end{enumerate}


\noindent\makebox[\linewidth]{\rule{\linewidth}{0.4pt}}
\section{\secfont Reports}
These are technical reports completed under contract for the sponsoring agency;
each report was reviewed by a technical advisory committee prior to publication.
\vspace{0.3cm}
\begin{enumerate}[left=0pt, itemsep=12pt, label=R\arabic*]
\item \textbf{Macfarlane, G.S.}, Lant, N.J.$^\dagger$, (2021). Estimation and Simulation of Daily Activity Patterns for Individuals Using Wheelchairs (No. UT-21.10). Utah Dept. of Transportation. Division of Research. \url{https://rosap.ntl.bts.gov/view/dot/54639/dot_54639_DS1.pdf}

\item Schultz, G. G., \textbf{Macfarlane, G.S.}, Wang, B.$^\dagger$, \& McCuen, S.$^*$ (2020). Evaluating the Quality of Signal Operations Using Signal Performance Measures (No. UT-20.08). Utah Dept. of Transportation. Division of Research. \url{https://rosap.ntl.bts.gov/view/dot/54639/dot_54639_DS1.pdf}

\item \textbf{Macfarlane, G.S.} \& Copley, M.J.$^*$ (2020). \textit{ A Synthesis of Passive Third-Party Data sets used for Transportation Planning.} (No. UT-20.20). Utah Dept. of Transportation. Division of Research. \url{https://rosap.ntl.bts.gov/view/dot/54890/dot_54890_DS1.pdf}

\item Zalewski, A., Sonenklar, D., Cohen, A., Kressner, J., \& \textbf{Macfarlane, G.S.} (2019). \textit{ Public Transit Rider Origin–Destination Survey Methods and Technologies}. TCRP Synthesis of Transit Practice 138. Transportation Research Board. \url{http://www.trb.org/Main/Blurbs/179008.aspx}

\item Miller, H., O'Kelly, M., Jaegal, Y., Bachman, W., Huntsinger, L., \& \textbf{Macfarlane, G.S.} (2017) Estimating External Travel Using Purchased Third-Party Data. Research Report 134877, the Ohio Department of Transportation, Office of Statewide Planning \& Research.

\item Cruz, J., \textbf{Macfarlane, G.S.}, Xu, Y., Rodgers, M.O., \& Guensler, R. (2015). Sustainable Transportation Curricula. National Center for Sustainable Transportation.
\end{enumerate}

\noindent\makebox[\linewidth]{\rule{\linewidth}{0.4pt}}
\section{\secfont Presentations}
This includes invited presentations to academic and non-academic audiences, as
well as presentations resulting from abstract-only submission. This includes
lectern sessions and posters.
\vspace{0.3cm}
\begin{enumerate}[left=0pt, itemsep=12pt, label=P\arabic*]
\item Turley Voulgaris, C., \textbf{Macfarlane, G.S.}, Kaylor, J., Su, T., Bauranov, A. (2021). Whose emissions are these anyway? Estimating vehicle miles traveled to account for site-level climate impacts. In \textit{Association of Collegiate Schools of Planning Annual Conference}. Lectern presentation. Miami, Florida.

\item \textbf{Macfarlane, G.S.}, Boyd, N., Taylor, J.E., \& Watkins, K.E. (2019). Modeling the impacts of park access on health outcomes: a choice-based accessibility approach. In \textit{ Greater and Greener 2019.}. Workshop presentation. Denver, Colorado.

\item Bernardin, V., Gallup, A., Lee, B., Johnson, C., \textbf{Macfarlane, G.S.}, Elgar, I., Wertman, R. (2019). How to be a Good Big Data Consumer. In \textit{ Transportation Planning Applications Conference}. Panel discussion. Portland, Oregon.

\item \textbf{Macfarlane, G.S.}, \& Kressner, J.D. (2018). Comparing the Daily Schedules in the NHTS from 2009 and 2017. In \textit{ National Household Travel Survey (NHTS) Data for Transportation Applications Workshop}. Poster. Washington, D.C.

\item \textbf{Macfarlane, G.S.}, Bettinardi, A.O., \& Donnelly, R. (2017). SWIMR: Visualizing complex longitudinal indicators for a statewide integrated land use and transport model in Oregon. In \textit{Transportation Planning Applications Conference}. Lectern presentation. Raleigh, North Carolina.

\item Boyd, N., \textbf{Macfarlane, G.S.}, Watkins, K.E., \& Ederer, D. (2017). Accessibility to urban parks and health outcomes on the neighborhood level. In \textit{ American Public Health Association Annual Meeting}. Poster. Atlanta, Georgia.

\item \textbf{Macfarlane, G.S.}, \& Kressner, J.D. (2017). Modeling automated vehicles with a passive data model. In \textit{ Transportation Planning Applications Conference}. Poster. Raleigh, North Carolina.

\item Kressner, J.D., \textbf{Macfarlane, G.S.}, Donnelly, R., \& Huntsinger, L.F. (2016). Using passive data to build an agile tour-based model: A case study in Asheville. In \textit{ Innovations in Travel Modeling Conference}. Lectern presentation. Denver, Colorado.

\item \textbf{Macfarlane, G.S.}, \& Kressner, J. D. (2016). Fusing Passive Data for Transportation Planning. In \textit{ Transportation Research Board Annual Meeting}. Poster. Washington, D.C.

\item \textbf{Macfarlane, G.S.}, \& Moreno-Cruz, J. (2015). The association between public transportation infrastructure and home price growth and stability. \textit{ In Transportation Research Board Annual Meeting}. Washington, D.C.

\item \textbf{Macfarlane, G.S.}, \& Garrow, L. A. (2012). Estimating a vehicle ownership model from targeted marketing data. In \textit{ Travel Surveys: Moving from Tradition to Practical Innovation}. Poster. Dallas, Texas.

\item Kressner, J.D., \& \textbf{Macfarlane}, G.S. (2012). Evaluating household credit reports as a replacement for episodic travel surveys. In \textit{ Transportation Research Board Annual Meeting}. Committee presentation. Washington, D.C.

\item \textbf{Macfarlane, G.S.}, Saito, M., \& Schultz, G.G. (2011). Are free right-turn channelized intersections performing as they should? In \textit{ Institute of Transportation Engineers Annual Meeting and Exhibit 2011}.
\end{enumerate}



%% ==========================================
\noindent\makebox[\linewidth]{\rule{\linewidth}{0.4pt}}
\section{\secfont External \\Funding}

As Principal Investigator, totalling \$305,000:
\vspace{0.3cm}
\begin{enumerate}[left=0pt, itemsep=12pt, label=\arabic*]
  \item {\textbf{Macfarlane, G.S.} \& Schultz, G.G. 2021. \textit{Optimizing Traffic Incident Management Deployment in Utah}. \$70,000, Utah Department of Transportation.}
  \item {\textbf{Macfarlane, G.S.}, Redelfs, A.H., \& Spruance, L.A. 2021. \textit{Equitable Access to Nutrition in Utah}. \$70,000, Utah Department of Transportation.}
  \item {\textbf{Macfarlane, G.S.} 2020. \textit{ Identifying Microtransit Service Areas
through Microsimulation}. \$20,000, Utah Department of Transportation}
  \item {\textbf{Macfarlane, G.S.} 2019. \textit{ A synthesis of passive third-party datasets
used for transportation planning}. \$25,000, Utah Department of Transportation}
  \item {\textbf{Macfarlane, G.S.} 2019. \textit{ Modeling the demand and operating
characteristics for wheelchair-accessible, on-demand mobility services}.
\$60,000, Utah Department of Transportation}
  \item {\textbf{Macfarlane, G.S.} 2019. \textit{ Evaluating the Systemic Redundancy of
Critical Highway Facilities}. \$60,000, Utah Department of Transportation}
\end{enumerate}

As Co-Principal Investigator, totalling \$1.16 million (\$235,000 to BYU):
\vspace{0.3cm}
\begin{enumerate}[left=0pt, itemsep=12pt, label=\arabic*]
  \item{Schultz, G.G. \& \textbf{Macfarlane, G.S.}. 2021. \textit{Analysis of
  performance measures of UDOT’s traffic incident management program: Phase
  III}. \$30,000. Utah Department of Transportation.}
  \item{Watkins, K.E. (PI), Hunter, M.S., Van Hentenryck, P., Peeta, S.,
  Brakewood, C., Erhardt, G.D., \& \textbf{Macfarlane, G.S.} 2020. \textit{
  T-SCORE: Transit Serving Communities Optimally, Responsibly, and Efficiently}.
  \$1,000,000, United States Department of Transportation.}
  \item{ Schultz, G.G. (PI), \textbf{Macfarlane, G.S.} 2020. \textit{ Evaluating Signal
  Performance Measures: a Longitudinal Analysis}. \$70,000, Utah Department of
  Transportation}
  \item{ Schultz, G.G. (PI), \textbf{Macfarlane, G.S.} 2019. \textit{ Evaluating ramp meter
delay in Utah}. \$65,000, Utah Department of Transportation}
\end{enumerate}

No unfunded proposals.

\section{\secfont{Internal Competitive Funding}}
Funded research:
\begin{itemize}
  \item {\textbf{Macfarlane, G.S.}, Guthrie, W.S., Mazzeo, B. 2021. \textit{ Measuring pavement smoothness from the perspective of e-scooters}. \$25,000, Mentored Research Grant, Brigham Young University.}
\end{itemize}

Unfunded proposals:
\begin{itemize}
  \item{\textbf{Macfarlane, G.S.}, Hooley, C., Redelfs, A., South, M. 2020 \textit{ Using Mobile Device Data to Measure Isolation and Mental Health}}.
  \$40,000, Brigham Young University Interdisciplinary Research Grant.
\end{itemize}

%% ============================================================================
\noindent\makebox[\linewidth]{\rule{\linewidth}{0.4pt}}
\section{\secfont Courses}

{\acc CCE 201: Sustainable Infrastructure}

\vspace{-.4cm}
The inter-related aspects of the different civil engineering disciplines of
environmental, geotechnical, structural, transportation, and water resources and
how they come together to develop an infrastructure system. Time value of money
and application to the infrastructure investment alternatives.

\begin{tabular}{cccc}
  \toprule
  Semester & Enrolled & Student Rating (Historical) & Average GPA\\
  \midrule
  Fall 2020 & 33 & 4.1 - 4.7 (4.1)& 3.41 \\
  Fall 2021 & & & \\
  \bottomrule
\end{tabular}

\vspace{.4cm}
{\acc CE 361: Introduction to Transportation Engineering}

\vspace{-.4cm}
Transportation systems characteristics, traffic engineering and operations,
transportation planning, geometric design, pavement design, transportation
safety, freight, public transport, sustainable transportation.

\begin{tabular}{cccc}
  \toprule
  Semester & Enrolled & Student Rating (Historical) & Average GPA\\
  \midrule
  Winter 2020 & 42 & 4.4 - 4.8 (4.4) & 3.13 \\
  Winter 2021 & 38 & 4.1 - 4.7 (4.4) & 3.21 \\
  \bottomrule

\end{tabular}

\vspace{.4cm}
{\acc CE 565: Urban Transporation Planning}

\vspace{-.4cm}
Characteristics of urban transportation planning and decision making, intermodal
transportation, land-use transportation interrelationships, transportation
demand modeling, site impact analysis, sustainable transportation, and livable
cities.

\begin{tabular}{cccc}
  \toprule
  Semester & Enrolled & Student Rating (Historical) & Average GPA\\
  \midrule
  Fall 2019 & 12 & 3.9 - 4.9 (4.4) & 3.41 \\
  Fall 2020 & 19 & 4.1 - 4.7 (4.4) & 3.46 \\
  Fall 2021 & & & \\
  \bottomrule

\end{tabular}

\vspace{.4cm}
{\acc CE 594R: Data Science for Engineers}

\vspace{-.4cm}
A first-semester graduate course in programming and data science techniques:
literate programming in Markdown and LaTeX, version control with git, data
manipulation and visualization with R, object-oriented programming with Java.

\begin{tabular}{cccc}
  \toprule
  Semester & Enrolled & Student Rating (Historical) & Average GPA\\
  \midrule
  Fall 2019 & 4 & 4.8 - 5.0 () & 3.85 \\
  Fall 2020 & 9 & 3.5 - 4.7 () & 3.81 \\
  Fall 2021 & & & \\
  \bottomrule

\end{tabular}



\vspace{.4cm}
{\acc CE 662: Transport Simulation and Analysis}

\vspace{-.4cm}
An advanced graduate course in traffic flow theory and simulation. Topics
include shock wave analysis, discrete event simulation of queues and daily
activity pattern choices, car following models, and traffic simulation.
Laboratory assignments use MATSim and PTV Vissim simulation softwares.


\begin{tabular}{cccc}
  \toprule
  Semester & Enrolled & Student Rating (Historical) & Average GPA\\
  \midrule
  Winter 2019 & 2 & 4.6 (4.3) & 3.70 \\
  Winter 2020 & 3 & 5.0 - 5.0 (4.4) & 3.00 \\
  \bottomrule
\end{tabular}


\vspace{.4cm}
{\acc CE 694R: Advanced Choice Modeling}

\vspace{-.4cm}
An advanced graduate course in discrete choice modeling. Theory of choice
models, including estimation and validation techniques. Mode choice models for
work and non-work trip purposes using multinomial and nested logit models.

\begin{tabular}{cccc}
  \toprule
  Semester & Enrolled & Student Rating (Historical) & Average GPA\\
  \midrule
  Winter 2021 & 5 & 4.0 - 5.0 () & 3.48 \\
  \bottomrule
\end{tabular}

% {\acc CEE 6622: Travel Demand Analysis}\hfill{Ga. Tech: Spring 2014}
%
% \vspace{-.4cm}
% This course teaches graduate students to develop and use urban travel
% demand models, including trip-based and activity-based modeling methods and
% experience with a modern practicing regional model.

%% ============================================================================
\noindent\makebox[\linewidth]{\rule{\linewidth}{0.4pt}}
\section{\secfont Graduate Mentoring}
Students mentored as graduate committee chair (6 total, 4 current):
\vspace{0.3cm}
\begin{enumerate}[left=0pt, itemsep=10pt, label=G\arabic*]
  \item Gillian Martin Riches. MS scheduled December 2022.
  \item Christopher Day. MS scheduled December 2022.
  \item Emma Stucki. MS scheduled December 2022.
  \item Natalie Gray. MS scheduled December 2022.
  \item Max Barnes, \textit{Resiliency of utah's road network: a logit-based approach.} MS scheduled December 2021.
  \item Nate Lant, \textit{Estimation and simulation of daily activity patterns for individuals using wheelchairs.} MS granted June 2021.
\end{enumerate}

Students mentored as graduate committee member (11 total, 4 current, 2 non-BYU):
\vspace{0.3cm}
\begin{enumerate}[left=0pt, itemsep=10pt, label=GM\arabic*]
  \item Tomas Barriga. MS scheduled December 2022.
  \item Tanner Daines. MS scheduled April 2022.
  \item Samantha Lau. MS scheduled April 2022.
  \item Wang Bangyu (Bruce). Ph.D.\ proposed May 2021.
  \item Logan Bennett, \textit{Analysis of benefits of an expansion to UDOT’s incident management program}. MS granted August 2021.
  \item Camille Lunt, \textit{Crash analysis methodology for segments of Utah highway}. MS granted April 2021.
  \item Chad Vickery, \textit{Quantifying the conditioning period for geogrid-reinforced aggregate base materials through cyclic loading}. MS granted August 2020.
  \item Michael Sheffield, \textit{Impacts of changing the transit signal priority requesting threshold on bus performance and general traffic: a sensitivity analysis}. MS granted June 2020.
  \item Michael Adamson, \textit{An analysis of decision boundaries for left-turn treatments}. MS granted April 2019.
  \item Nico Boyd, \textit{Accessibility to urban parks and health outcomes on the neighborhood level}. MS granted August 2018 (at Georgia Tech).
  \item Zhang Bingling, \textit{Friendship influences on air travel: a social autoregressive analysis}. MS granted August 2014 (at Georgia Tech).
\end{enumerate}


\noindent\makebox[\linewidth]{\rule{\linewidth}{0.4pt}}
\section{\secfont Undergraduate Mentoring}

Students mentored on funded research projects (15 total):
\vspace{0.2cm}
\begin{enumerate}[left=0pt, itemsep = 10pt, label=\arabic*]
  \item Dylan Apelu, undergraduate research assistant in e-scooters and pavements (2021 - ).
  \item Hayden Atchley, undergraduate research assistant in demand microsimulation (2020 - )
  \item Nicole Anderson, undergraduate research assistant in e-scooters and pavements (2021 - ).
  \item Liv Neeley, undergraduate research assistant in e-scooters and pavements (2021 - ).
  \item Kaeli Monahan, undergraduate research assistant in community resources and passive data (2020 - ).
  \item Corey Ward, undergraduate research assistant in ramp meter evaluation, jointly mentored with Grant Schultz (2020 - ).
  \item Michael Copley, undergraduate research assistant in third-party passive data (2018 - ).
  \item Christopher Day, undergraduate research assistant in demand microsimulation (2020 - 2021). Now MS student at BYU.
  \item Emma Stucki, undergraduate research assistant in community resources (2020 - 2021). Now MS student at BYU.
  \item Gillian Martin Riches, undergraduate research assistant in community resources (2020 - 2021). Now MS student at BYU.
  \item Natalie Gray, undergraduate research assistant in network resiliency (2019 - 2021). Now MS student at BYU.
  \item Christian Hunter, undergraduate research assistant in demand microsimulation (2018 - 2019). Now MS student at University  of Texas at Austin.
  \item Christian Vanderhoeven, undergraduate research assistant in demand microsimulation (2019). Now MS student at University of Washington.
  \item Emily Andrus, undergraduate research assistant in signal peformance data, jointly mentored with Grant Schultz (2019). Now working as an engineering consultant.
  \item Sabrina McCuen, undergraduate research assistant in signal peformance data, jointly mentored with Grant Schultz (2019 - 2020). Now working as an engineering consultant.
\end{enumerate}


Students mentored as civil engineering capstone team advisor (15 total):
\vspace{0.2cm}
\begin{description}
  \item[2021-2022] Carbon footprint of daily commuting to BYU campus. Sponsored by BYU Sustainability Office. Students: Nicole Anderson, Hayden Atchley, Kyle Leatham, and Daniel Jarvis.
  \item[2020-2021] Forecasting demand for future FrontRunner scenarios. Sponsored by Utah Transit Authority. Students: Gillian Martin Riches, Tomas Barriga, Landon Pratt, and Cole Larsen.
  \item[2019-2020] UTA microtransit pilot evaluation. Sponsored by Utah Transit Authority. Students: Christian Hunter, Austin Martinez, and Elizabeth Smith.
  \item[2018-2019] Demand for wheelchair-accessible vehicles. Sponsored by Utah Transit Authority. Students: Nate Lant, Byron Yates, Cody Irons, and Matthew Strong.
\end{description}


% \section{\secfont Professional \\Experience}
% {\acc Singapore Mission of the Church of Jesus Christ of Latter-day Saints}
%
% \vspace{-.3cm}
% \textit{ Missionary} \hfill {June 2004 - June 2006}\\
% Served as volunteer cleric in Singapore, Malaysia, and Sri Lanka.



%% ============================================================================
\noindent\makebox[\linewidth]{\rule{\linewidth}{0.4pt}}
\section{\secfont Awards and\\ Honors}
\begin{description}
\item[\acc Dwight David Eisenhower Graduate Fellowship] Full doctoral funding from
the Federal Highways Administration 2011-2013, one of five awards nationally.
Awarded supplemental grant in 2013.
\item[\acc Eno Center for Transportation Leadership Development Conference]
Participated in the 2012 program; nominated by the Ivan Allen, Jr.\ College of
Liberal Arts at Georgia Tech.
\item[\acc Parsons Brinckerhoff - Jim Lammie Engineering Scholarship] Awarded
 to the top engineer in the 2011 American Public Transportation Foundation (APTF)
competition. Sponsored by Mike Allegra, general manager of the Utah Transit
Authority. Renewed in 2012.
% Lesser awards (comment out for space saving)
\item[\acc Gordon W. Schultz Graduate Fellowship] Given to the Georgia Tech
student in travel demand modeling who exhibits innovation, problem-solving, and
practical application.
\item[\acc National Science Foundation Graduate Fellowship Program] Honorable
Mention in 2011 and 2012, as a first- and second-year graduate student.
\item[\acc Jim McGee Memorial Scholarship] Cash award from the Georgia chapter of the
American Society of Highway Engineers, one of two awards in 2011.
\item[\acc Georgia Department of Transportation Scholarship] One of ten cash awards in
2010 to students from the southeastern United States.
\item[\acc Office of Research and Creative Activities (ORCA) Grant] Competitive
research grant to survey Chinese transportation planning practices, one of
several undergraduate research awards from Brigham Young University.
\item[\acc Freeman-Asia Award] Grant to study Chinese finance and globalized
engineering at Nanjing University in the People's Republic of China from the
Institute for International Education.
\end{description}

%% ============================================================================
\noindent\makebox[\linewidth]{\rule{\linewidth}{0.4pt}}
\section{\secfont External \\ Citizenship}

Registered professional engineer in North Carolina, license \#044518

Transportation Research Board of the National Acadmies of Science:

\begin{itemize}
  \item AEP50: Travel Demand Forecasting Member of the committee (2019 --- ) on
  travel demand forecasting. Chair of the travel forecasting resources
  subcommittee and editor of \url{tfresource.org}.
  \item AMS20: Economics and Land Development Member of the committee (2014
  --- ). formerly standing committee on transportation and land use.
  \item Young Members Council (2019 --- 2021). Planning and Environment subcommittee
  chair.
\end{itemize}


Reviewer for the following journals:

\begin{itemize}
  \item Transportation Research Part A: Policy and Practice
  \item Transportation Research Record
  \item Environment and Planning B: Urban Analytics and City Science
  \item International Journal of Sustainable Transport
  \item Journal of Transportation
\end{itemize}

Member of the following professional organizations:

\begin{itemize}
  \item Zephyr Foundation (2020 - ).
  \item Institute of Transportation Engineers (2009-2013, 2018-2020)
  \item Tau Beta Pi (Utah $\beta$ '09).
  \item Young Professionals in Transportation (2013-2018); organizing co-chair of
Triangle NC chapter.
  \item American Public Transportation Association scholar task force (2011 - 2013).
\end{itemize}

\section{\secfont Media}

McCann, A. (2021). Best and worst cities to drive in. \textit{WalletHub}. Quoted expert opinion. August 31, 2021. \url{https://wallethub.com/edu/best-worst-cities-to-drive-in/13964#expert=Gregory_Macfarlane}

\textbf{Macfarlane, G.S.}. (2020). No, Utah County does not have to choose between preservation and growth. \textit{Deseret News}. Guest Opinion, August 21, 2020. \url{https://www.deseret.com/opinion/2020/8/21/21376479/}

\noindent\makebox[\linewidth]{\rule{\linewidth}{0.4pt}}
\section{\secfont Internal \\ Citizenship}
\begin{description}
  \item[Department honors coordinator] (2019 --- ). Encourage students to participate
  in the honors program, and participate on honors thesis committees in the
  department.
  \item[Department undergraduate committee] (2021 --- ).
  \item[Department faculty development and capital improvement committee] (2018 --- 2021).
\end{description}


%% ==========================================

% \noindent\makebox[\linewidth]{\rule{\linewidth}{0.4pt}}
% \section{\secfont Skills}
% Significant computer software and programming ability, including expert skills
% in R, \LaTeX, and git. Also experienced with Java, C, Python, QGIS,
% Cube/Voyager,  Matlab, PTV Vissim / Visum, TransCAD (with GISDK).


%\clearpage
%\thispagestyle{empty}
%\section{\secfont References}
%\textit{ Rick Donnelly, Ph.D., AICP} \\
%6100 Uptown Boulevard NE, Suite 700, Albuquerque, NM 87110 \\
%DonnellyR@pbworld.com 1.505.878.6524
%
%\textit{ Leta F. Huntsinger, Ph.D., PE} \\
%434 Fayetteville Drive, Suite 1500, Raleigh NC 27601 \\
%Huntsinger@pbworld.com 1.919.836.4086
%
%%\textit{ Laurie A. Garrow, Ph.D.} \\
%%790 Atlantic Drive, Atlanta GA 30332-0355\\
%%laurie.garrow@ce.gatech.edu 1.404.385.6634
%
%\textit{ Juan Moreno-Cruz, Ph.D.} \\
%221 Bobby Dodd Way, Atlanta GA 30332\\
%juan.moreno-cruz@econ.gatech.edu 1.404.385.1100
%
%\textit{ Kari E. Watkins, Ph.D., P.E.} \\
%790 Atlantic Drive, Atlanta GA 30332-0355\\
%kari.watkins@ce.gatech.edu 1.206.250.4415
%
%%\textit{ Patricia L. Mokhtarian, Ph.D.} \\
%%790 Atlantic Drive, Atlanta GA 30332-0355\\
%%patmokh@ce.gatech.edu 1.404.385.1443
%
%%\textit{ Patrick S. McCarthy, Ph.D.} \\
%%221 Bobby Dodd Way, Atlanta GA 30332\\
%%mccarthy@gatech.edu 1.404.894.4914
%
%%\textit{ Mitsuru Saito, Ph.D., P.E.} \\
%368J Clyde Building, Provo UT 84602\\
%msaito@byu.edu 1.801.422.6326
\end{resume}
%\vfill\moveleft.25\hoffset\centerline{References are available upon request.}

\clearpage
\bibliographystyle{unsrt}
\bibliography{library}


\end{document}
